\documentclass[main.tex]{subfiles}
\begin{document}
%%\chapter{Tableaux et graphiques complémentaires}
\newpage
========= DEBUT ANNEXE TABLEAUX ET GRAPHIQUES =====

%%\section{Tableaux d'optimisation sur les niveaux de gris}
%%%%%%% Cout0
%\DTLloaddb{optim2_0grey}{../data/cout_0/optim2.csv}
%\DTLloaddb[noheader]{optim2_0leg}{../data/cout_0/optim2_leg.csv}
%\DTLloaddb{optim2_0stat}{../data/cout_0/optim2_stat.csv}
%\DTLsetheader{optim2_0leg}{Column1}{}
%\DTLsetheader{optim2_0leg}{Column2}{}
%\begin{table}[htbp]
%\footnotesize
%\hspace{\retraittableau} %%% Le tableau depasse sur les marges !
%\begin{tabular}{|c|c|c|c|c|}
%\hline
%%%\hhline{|>{\arrayrulecolor{white}}->{\arrayrulecolor{black}}|-|-|}
%\rowcolor{gray!70}
%& \multicolumn{4}{c|}{ \cellcolor{gray!70} \bfseries  Algorithme d'optimisation} \\
%\hhline{|>{\arrayrulecolor{gray!70}}->{\arrayrulecolor{black}}|-|-|-|-|}
%\rowcolor{gray!70}
%& \bfseries SLSQP
%& \bfseries GC 
%& \bfseries Neldear-Mead 
%& \bfseries BFGS \\
%\rowcolor{gray!70}
%\multirow{-3}{\firstcolwidth}{\scriptsize \bfseries \centering Scanners choisis pour l'optimisation}
%& $\tau_N, \qquad \tau_P$
%& $\tau_N, \qquad \tau_P$
%& $\tau_N, \qquad \tau_P$
%& $\tau_N, \qquad \tau_P$
%\DTLforeach*{optim2_0grey}{%
%\scan=scan,\NM=NM,\BFGS=BFGS,\col=SLSQP,\CG=CG,
%\errNM=errNM,\errBFGS=errBFGS,\errcol=errSLSQP,\errCG=errCG}{%
%\\
%\DTLifoddrow{\rowcolor{white}}{\rowcolor{gray!40}}%
%\scan & \begin{tabular}{c}
%\col \\ \errcol
%\end{tabular} & \begin{tabular}{c}
%\CG \\ \errCG
%\end{tabular} & \begin{tabular}{c}
%\NM \\ \errNM
%\end{tabular} & \begin{tabular}{c}
%\BFGS \\ \errBFGS
%\end{tabular} 
%}%
%\DTLforeach*{optim2_0stat}{%
%\NM=NM,\BFGS=BFGS,\col=SLSQP,\CG=CG}{%
%\\ \hline \hline %\DTLifoddrow{\rowcolor{white}}{\rowcolor{gray!40}}%
%Moyenne : & \col & \CG & \NM & \BFGS  }
%\\ \hline
%\end{tabular}
%\caption{\label{tab:optim2gris0}Tableau récapitulatif des optimisations réalisées sur 2 niveaux de gris, $\tau_S$ fixé à 197, PAS DE PENALISATION.}
%\end{table}
%\DTLcleardb{optim2_0leg}
%\DTLcleardb{optim2_0stat}
%\DTLcleardb{optim2_0stat}


%%%%%% Cout1
\DTLloaddb{optim2_1grey}{../data/cout_1/optim2.csv}
\DTLloaddb[noheader]{optim2_1leg}{../data/cout_1/optim2_leg.csv}
\DTLloaddb{optim2_1stat}{../data/cout_1/optim2_stat.csv}
\DTLsetheader{optim2_1leg}{Column1}{}
\DTLsetheader{optim2_1leg}{Column2}{}
\begin{table}[htbp]
%\vspace{-35mm}
\footnotesize
\hspace{\retraittableau} %%% Le tableau depasse sur les marges !
\begin{tabular}{|c|c|c|c|c|}
\hline
%%\hhline{|>{\arrayrulecolor{white}}->{\arrayrulecolor{black}}|-|-|}
\rowcolor{gray!70}
& \multicolumn{4}{c|}{ \cellcolor{gray!70} \bfseries  Algorithme d'optimisation} \\
\hhline{|>{\arrayrulecolor{gray!70}}->{\arrayrulecolor{black}}|-|-|-|-|}
\rowcolor{gray!70}
& \bfseries SLSQP
& \bfseries GC 
& \bfseries Neldear-Mead 
& \bfseries BFGS \\
\rowcolor{gray!70}
\multirow{-3}{\firstcolwidth}{\scriptsize \bfseries \centering Scanners choisis pour l'optimisation}
& $\tau_N, \qquad \tau_P$
& $\tau_N, \qquad \tau_P$
& $\tau_N, \qquad \tau_P$
& $\tau_N, \qquad \tau_P$
\DTLforeach*{optim2_1grey}{%
\scan=scan,\NM=NM,\BFGS=BFGS,\col=SLSQP,\CG=CG,
\errNM=errNM,\errBFGS=errBFGS,\errcol=errSLSQP,\errCG=errCG}{%
\\
\DTLifoddrow{\rowcolor{white}}{\rowcolor{gray!40}}%
\scan & \begin{tabular}{c}
\col \\ \errcol
\end{tabular} & \begin{tabular}{c}
\CG \\ \errCG
\end{tabular} & \begin{tabular}{c}
\NM \\ \errNM
\end{tabular} & \begin{tabular}{c}
\BFGS \\ \errBFGS
\end{tabular} 
}%
\DTLforeach*{optim2_1stat}{%
\NM=NM,\BFGS=BFGS,\col=SLSQP,\CG=CG}{%
\\ \hline \hline %\DTLifoddrow{\rowcolor{white}}{\rowcolor{gray!40}}%
Moyenne : & \col & \CG & \NM & \BFGS  }
\\ \hline
\end{tabular}
%\centering
%\begin{tabular}{cc}
%\DTLdisplaydb{optim2_leg}
%\end{tabular}
\caption{\label{tab:optim2gris}Tableau récapitulatif des optimisations réalisées sur 2 niveaux de gris, $\tau_S$ fixé à 197, avec un créneau comme pénalisation de l'intervalle.}
\vspace{-4cm}
\end{table}
\DTLcleardb{optim2_1leg}
\DTLcleardb{optim2_1stat}
\DTLcleardb{optim2_1stat}


%%%%%% Cout2
\DTLloaddb{optim2_2grey}{../data/cout_2/optim2.csv}
\DTLloaddb[noheader]{optim2_2leg}{../data/cout_2/optim2_leg.csv}
\DTLloaddb{optim2_2stat}{../data/cout_2/optim2_stat.csv}
\DTLsetheader{optim2_2leg}{Column1}{}
\DTLsetheader{optim2_2leg}{Column2}{}
\begin{table}[htbp]
\footnotesize
\hspace{\retraittableau} %%% Le tableau depasse sur les marges !
\begin{tabular}{|c|c|c|c|c|}
\hline
\rowcolor{gray!70}
& \multicolumn{4}{c|}{ \cellcolor{gray!70} \bfseries  Algorithme d'optimisation} \\
\hhline{|>{\arrayrulecolor{gray!70}}->{\arrayrulecolor{black}}|-|-|-|-|}
\rowcolor{gray!70}
& \bfseries SLSQP
& \bfseries GC 
& \bfseries Neldear-Mead 
& \bfseries BFGS \\
\rowcolor{gray!70}
\multirow{-3}{\firstcolwidth}{\scriptsize \bfseries \centering Scanners choisis pour l'optimisation}
& $\tau_N, \qquad \tau_P$
& $\tau_N, \qquad \tau_P$
& $\tau_N, \qquad \tau_P$
& $\tau_N, \qquad \tau_P$
\DTLforeach*{optim2_2grey}{%
\scan=scan,\NM=NM,\BFGS=BFGS,\col=SLSQP,\CG=CG,
\errNM=errNM,\errBFGS=errBFGS,\errcol=errSLSQP,\errCG=errCG}{%
\\
\DTLifoddrow{\rowcolor{white}}{\rowcolor{gray!40}}%
\scan & \begin{tabular}{c}
\col \\ \errcol
\end{tabular} & \begin{tabular}{c}
\CG \\ \errCG
\end{tabular} & \begin{tabular}{c}
\NM \\ \errNM
\end{tabular} & \begin{tabular}{c}
\BFGS \\ \errBFGS
\end{tabular} 
}%
\DTLforeach*{optim2_2stat}{%
\NM=NM,\BFGS=BFGS,\col=SLSQP,\CG=CG}{%
\\ \hline \hline %\DTLifoddrow{\rowcolor{white}}{\rowcolor{gray!40}}%
Moyenne : & \col & \CG & \NM & \BFGS  }
\\ \hline
\end{tabular}
\caption{\label{tab:optim2gris_pen_quad}Tableau récapitulatif des optimisations réalisées sur 2 niveaux de gris, $\tau_S$ fixé à 197, avec pénalisation quadratique \eqref{eq:penalisation_quad}.}
\end{table}


\end{document}