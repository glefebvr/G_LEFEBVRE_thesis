\documentclass[main.tex]{subfiles}
\pagestyle{main}

\begin{document}
\chapter{Un modèle mathématique qui reproduit la croissance tumorale \label{chap:modele_EDP}}

%\mylettrine{N}{ous} allons ici présenter le cheminement qui, depuis les mécanismes biologiques et cliniques, a conduit à un modèle mathématique. 

\mylettrine{C}{e} chapitre, ainsi qu'une partie du chapitre suivant, %(la partie sur le \twinweno) 
ont  fait l'objet d'une publication~\cite{lefebvre:hal-01089452} (actuellement en review) et d'un poster (présenté en Annexe, dans la Figure~\ref{fig:poster}, page~\pageref{fig:poster}). %Nous allons ici garder la structure de l'article, en commençant par présenter les principaux résultats. 
Après avoir présenté la problématique, les principaux résultats obtenus seront décrits.  
La construction du modèle sera ensuite détaillée. Nous discuterons enfin plus précisément, sur les résultats numériques apportés par notre modèle et les diverses informations que l'on peut en tirer.

\section{Problématique}
Le but de ce chapitre est de construire un modèle de croissance tumorale adapté aux métastases hépatiques de GIST et à leurs traitements. Un comportement typique est celui de \Nber dont les données d'imagerie sont présentées en Annexe~\ref{chap:anx_img_complement}, sur les Figures~\ref{fig:nber_complete_scan} et~\ref{fig:contourageNber}. Le schéma typique est le suivant~: à la détection d'une métastase (pour \Nber, à l'examen n°1, jour~119) le patient reçoit de l'imitanib (thérapie ciblée). La réponse est positive~: la tumeur décroît et est contrôlée jusqu'au jour~867 (examen n°11) où l'on constate un échappement au traitement. Celui-ci est alors modifié~: l'imatinib est  remplacé par le sunitinib, thérapie ciblée ayant notamment des effets antiangiogéniques. Ce traitement permet une nouvelle fois le contrôle de la lésion jusqu'à une nouvelle rechute constatée jour~1179 (examen n°11). Ce comportement est typique chez les métastases hépatiques de GIST et c'est celui-ci que nous souhaitons modéliser.

\section{Principaux résultats}
Sur la Figure~\ref{fig_nber} est présentée l'évolution d'une métastase hépatique de GIST de \Nber~à travers une séquence de scanners (sur la partie gauche de la figure) et l'aire tumorale (sur la partie droite de la figure). 
%d'un patient spécifique, appelé \Nber. 
Sur chaque scanner de \Nber, nous avons contouré la région la plus sombre, qui correspond principalement à la zone nécrosée.  
Pendant l'évolution tumorale, on peut voir que l'hétérogénéité tumorale change (par exemple sur la Figure~\ref{fig_hen_spatial_2} la tumeur est homogène tandis que sur la Figure~\ref{fig_hen_spatial_6}, elle est hétérogène). Nous avons également retranscrit cela dans le graphique de l'évolution de l'aire tumorale~: nous différencions les points correspondant à une tumeur homogène (cercles pleins) des points correspondant à une tumeur hétérogène.

\begin{figure}[t]
\captionsetup[subfigure]{margin={7pt,7pt}}
\begin{tabular}{@{\hspace{-1mm}}ccc}
\subfloat[16 sept. 2008 -- Jour 119]{\label{fig_hen_spatial_1}
\includegraphics[width=0.2\textwidth]{scan_henbert/2008_09_16(2).jpg}
}
&
\subfloat[30 juin 2009 -- Jour 406]{\label{fig_hen_spatial_2}
\includegraphics[width=0.2\textwidth]{scan_henbert/2009_06_30(2).jpg}
}&
\\
\subfloat[5 juill. 2010  -- Jour 776]{\label{fig_hen_spatial_3}
\includegraphics[width=0.2\textwidth]{scan_henbert/2010_07_05(3).jpg}
}&
\subfloat[25 oct. 2010 -- Jour 888]{\label{fig_hen_spatial_4}
\includegraphics[width=0.2\textwidth]{scan_henbert/2010_10_25.jpg}
}&\\
\subfloat[7 janv. 2011 -- Jour 962]{\label{fig_hen_spatial_5}
\includegraphics[width=0.2\textwidth]{scan_henbert/2011_01_07(2).jpg}
}&
\subfloat[10 juin 2011 -- Jour 1116]{\label{fig_hen_spatial_6}
\includegraphics[width=0.2\textwidth]{scan_henbert/2011_06_10.jpg}
}&
\multirow{-3}{*}[107mm]{
 \subfloat[Aire tumorale. Chaque point représente l'aire de la tumeur mesurée sur les scanners et la ligne représente notre résultat numérique. \newline
Les lettres font référence aux scanners montrés sur la gauche. Les symboles 
 $\oplus$ représentent les tumeurs hétérogènes, {\large$\bullet$} les tumeurs plutôt homogènes et {\large$\circ$} les tumeurs pour lesquelles il est difficile de détecter sur le scanner un aspect hétérogène ou homogène dominant.  
Les valeurs des paramètres utilisés dans la simulation numérique sont données dans la  Table~\ref{table_param}.]{\label{fig:fit_area_henbert}
 \includegraphics[width=0.5\textwidth]{fit_henbert_form3/henbert_area.png}
 }}
\end{tabular}
\caption{\label{fig_nber} Evolution spatiale de la métastase hépatique de \Nber sur une série de scanners. Ici n'est présenté qu'une sélection de scanners~: l'ensemble complet est présenté dans l'annexe~\ref{chap:anx_img_complement}.}
\end{figure}



%Curieusement, juste a
Avant les recroissances tumorales au Jour~776 (Figure~\ref{fig_hen_spatial_3})
et au Jour~1116 (Figure~\ref{fig_hen_spatial_6}), une couronne plus claire que le c\oe{}ur sombre apparaît, alors que la réponse au traitement se traduit par un assombrissement de la tumeur (\cf Figure~\ref{fig_hen_spatial_2} et~\ref{fig_hen_spatial_5}). 
De telles étapes successives de tumeurs homogènes/hétérogènes sont particulièrement prononcées dans les métastases hépatiques de GIST et notre but est de fournir une explication à ce type de comportement.


Le premier résultat de ces travaux est qu'il est possible de décrire l'évolution tumorale en terme d'aire tumorale comparable aux mesures réalisées sur les scanners, comme présenté sur la  Figure~\ref{fig:fit_area_henbert}; la ligne continue représentant notre résultat numérique. Il est important de noter que nous ne fournissons pas un modèle 1D qui décrit le volume tumoral. Nous traitons un modèle complexe d'EDPs non linéaires, qui est phénoménologique et qui décrit le comportement des cellules cancéreuses en fonction du temps et de l'espace. Par conséquent, le premier fit de l'aire tumorale n'est pas un résultat trivial.


Le second résultat est que notre modèle apporte de nouvelles informations sur la structure de la tumeur, ce qui semble corroboré par les scanners. En effet, comme présenté dans la suite, nous relions l'hétérogénéité tumorale à une augmentation de l'activité cellulaire signifiant qu'un phénotype résistant émerge dans la région plus claire. Selon notre modèle, de tels comportements peuvent être constatés avant la rechute effective, alors que le critère RECIST n'a pas changé. Par exemple, sur la Figure~\ref{fig:compare_spatial_nber}, 
nous comparons la structure de la tumeur à deux temps différents: l'hétérogénéité de la tumeur semble bien capturée par les simulations numériques. Ainsi, nos travaux peuvent être vu comme un premier pas dans le développement de nouveaux outils pour évaluer la réponse des tumeurs aux traitements  basés sur des inhibiteurs de tyrosine kinase.

\begin{figure}[htpb]
\vspace{-1cm}
\subfloat[Jour 406]{
\resizebox{.5\textwidth}{!}{
\raisebox{0.045\textwidth}{
\includegraphics[width=0.19\textwidth]{scan_henbert/2009_06_30(2).jpg}}
\includegraphics[width=0.3\textwidth]{fit_henbert_form3/vue_scan/vue_scan32.png}
}}
\subfloat[Jour 776]{
\resizebox{.5\textwidth}{!}{
\raisebox{0.045\textwidth}{
\includegraphics[width=0.19\textwidth]{scan_henbert/2010_07_05(3).jpg}}
\includegraphics[width=0.3\textwidth]{fit_henbert_form3/vue_scan/vue_scan60.png}
}}
\caption{\label{fig:compare_spatial_nber}
Comparaison entre les scanners de \Nber et les simulations numériques.
}
\end{figure}
%\vspace{-3cm}
\section{Description du modèle}

Notre modèle est un système d'EDPs similaire au modèle de 
Bresch {\it et al.}~\cite{Bresch2010}. 
%Grosso modo, l
La tumeur est décrite grâce à des densités de cellules proliférantes et une densité de cellules cancéreuses nécrosées notées~$P$ et~$N$ respectivement. 
La prolifération cellulaire conduit à une augmentation de l'aire tumorale causée par une surpression (modélisée via~$\Pi$) dont le gradient~$\vit$ transporte les cellules saines environnantes~$S$ loin du c\oe{}ur de la tumeur. 
Les cellules situées au centre de la tumeur se nécrosent à cause du manque d'oxygène, sauf si l'angiogenèse intervient pour leur apporter les provisions nutritives nécessaires.
L'angiogenèse et l'apport en nutriments sont pris en compte dans notre modèle via une description simpliste similaire à celle de~\cite{Fathallah2013,Billy2009545} (on pourra se référer à~\cite{macklin2009multiscale} pour un modèle plus complet de l'angiogenèse)~: comme la vascularisation gouverne la concentration de nutriments parvenant à la tumeur, nous introduisons une variable~$M$ 
qui décrit à la fois la vascularisation, la néovascularisation et les nutriments apportés à la tumeur grâce à une équation d'advection-diffusion. 
Nous introduisons également les effets liés aux facteurs de croissance au travers de la variable~$\xi$ qui module~$M$. L'ensemble des quantités utilisées dans notre modèle est résumé dans la Table~\ref{table_variable}.

 \begin{table}[t]
 %\begin{threeparttable}[b]
 \centering
 \rowcolors{2}{white}{\grisclair}
 {\footnotesize%%\smaller[0.5]
 %%\small
 %%\scriptsize 
 \begin{tabularx}{\textwidth}{|l|X|l|}
 \hline
 \rowcolor[gray]{0.7} 
 Nom & Signification  & Unité \\
 \hline
 $P_1(t,\vecx)$ & Fraction de cellules proliférantes qui sont sensibles 
 aux traitements~$\TI$ et~$\TS$ & -  \\
% à~$\TI$ et à~$\TS$ & -  \\
 \hline
 $P_2(t,\vecx)$ & Fraction de cellules proliférantes qui sont résistantes à~$\TI$ et sensibles à~$\TS$  & -  \\
 \hline
 $P_3(t,\vecx)$ & Fraction de cellules proliférantes qui sont résistantes à~$\TI$ et à~$\TS$  & -  \\
 \hline
 $P(t,\vecx)$ & Fraction de cellules cancéreuses proliférantes (égale à~$P_1+P_2+P_3$) & -  \\
 \hline
 $N(t,\vecx)$ & Fraction de cellules nécrosées & - \\
 \hline
 $S(t,\vecx)$ & Fraction de cellules saines & - \\
 \hline
 $M(t,\vecx)$ & Fraction de nutriments // Vascularisation & - \\
 \hline
 $\xi(t)$ &Vitesse moyenne de transport des nutriments en direction de la tumeur & $cm.d^{-1}$ \\
 \hline
 $\vit(t,\vecx)$ &Vitesse du mouvement passif de la tumeur sous l'effet de la pression& $cm.d^{-1}$ \\
 \hline
 $\Pi(t,\vecx)$ & Pression du milieu\protect\footnotemark
% \tnote{1}
%% \tablefootnote{
%%L'unité de masse dans la pression~$\Pi$ et la perméabilité~$k$ n'ont pas d'importance. Seule la quantité~$k\nabla\Pi$ est pertinente et ce terme est homogène à~$cm.d^{-1}$. 
%%Ainsi~$k$ et~$\Pi$ doivent juste avoir la même unité arbitraire de masse.}
 & $kg.cm^{-1}.d^{-2}$\\
 \hline
 \end{tabularx}
 }
 \caption{\label{table_variable}Liste des quantités calculées par le modèle
 -- $d=\textrm{jour (day)}$ } 
%%\begin{flushleft}  \rule{5cm}{1pt}  \end{flushleft}
%\footnoterule 
% \begin{tablenotes}
%\item [1] L'unité de masse dans la pression~$\Pi$ et la perméabilité~$k$ n'ont pas d'importance. Seule la quantité~$k\nabla\Pi$ est pertinente et ce terme est homogène à~$cm.d^{-1}$. 
%Ainsi~$k$ et~$\Pi$ doivent juste avoir la même unité arbitraire de masse
%\end{tablenotes}
% \end{threeparttable}
 \end{table}
\footnotetext{L'unité de masse dans la pression~$\Pi$ et la perméabilité~$k$ n'ont pas d'importance. Seule la quantité~$k\nabla\Pi$ est pertinente et ce terme est homogène à~$cm.d^{-1}$. 
Ainsi~$k$ et~$\Pi$ doivent juste avoir la même unité arbitraire de masse.}


La principale nouveauté réside en la modélisation des traitements. Deux traitements sont considérés (on peut voir dans \cite{lorz2013,lorz2013populational}, les récents travaux de Lorz~\etal\ pour des modèles plus élaborés sur les résistances aux traitements). 
Le premier est un inhibiteur de tyrosine kinase (imatinib), %, comme l'imatinib, 
qui a un effet cytotoxique 
sur les cellules proliférantes. Le second est inhibiteur multi-kinases, comme le sunitinib ou le sorafenib, qui a un effet à la fois cytotoxique et antiangiogénique. En d'autres termes, en plus de l'effet cytotoxique, il bloque la production de facteurs de croissance, comme le facteur de croissance vasculaire endothélial (VEGF) et ainsi réduit la quantité de nutriments apportée à la tumeur. 
Il est bien connu que les médicaments cytotoxiques n'ont pas un impact similaire sur toutes les cellules cancéreuses métastatiques car des phénotypes résistants peuvent apparaître dans la population de cellules proliférantes. De plus, les cellules cancéreuses peuvent résister différemment à l'hypoxie. Par conséquent nous découpons, comme dans~\cite{Bresch2009}, la densité~$P$ de cellules proliférantes en 3 sous-populations~$P_1$, $P_2$ et~$P_3$, telles  que~$P=P_1+P_2+P_3$, où
\begin{myitemize}
\item $P_1$ représente la fraction de cellules proliférantes qui sont sensibles au premier traitement~$\TI$, basé sur la molécule d'imatinib, et sensible aussi au second traitement~$\TS$, à base de sunitinib ou de sorafenib, qui a un effet à la fois cytoxique et antiangiogénique. 
\item $P_2$ désigne la densité de cellules proliférantes qui sont résistantes à~$\TI$ et sensibles à~$\TS$,
\item $P_3$ représente les cellules proliférantes résistantes aux deux traitements. 
\end{myitemize}



Il est important de noter que notre but n'est pas de décrire l'évolution de la tumeur depuis le tout début du GIST, mais de nous focaliser sur l'évolution des métastases situées dans le foie. 
Par conséquent, selon les observations cliniques, il semble pertinent de considérer que les trois sous-populations sont déjà présentes lorsqu'une métastase est détectée.

\subsection{Le modèle mathématique}

Soit~$\Omega$ le domaine d'intérêt, où la tumeur croît, et soit~$\partial\Omega$ son bord, qui est supposé aussi régulier que nécessaire. Le domaine~$\Omega$ est supposé strictement inclus dans le tissu --le foie-- qui est plus grand. De plus, nous supposons que la tumeur n'atteint pas le bord~$\partial\Omega$.

\subsubsection{EDPs sur les cellules proliférantes, nécrosées et saines}

La densité de cellules proliférantes est gouvernée par l'équation de transport suivante~:
%%{%%\scriptsize
\begin{align}
%\medmuskip=1mu
%\thinmuskip=0mu
%\thickmuskip=0mu
\mkern-8mu \partial_t P_1 & + \dive(\vit P_1) = \big(\gammapp(M)- \gammapd(M)\big)P_1 \nonumber \\
 & \hspace{4cm}  - \big(\muI \chiI (t)+\muS \chiS(t) \big)(1+M)P_1 & \textrm{ dans } \Omega, \label{eq_P1}\\
\mkern-8mu \partial_t P_2 &+ \dive(\vit P_2) = \big(\gammapp(M)- \gammapd(M)\big)P_2 - \muS \chiS(t) (1+M)P_2 & \textrm{ dans }  \Omega, \label{eq_P2}\\
\mkern-8mu \partial_t P_3 &+ \dive(\vit P_3) = (\gammapp(M) - \gammapd(M))P_3 & \textrm{ dans }  \Omega, \label{eq_P3}
\end{align}%%}
où~$\chii(t)=\mathds{1}_{[\Tini{i},\rm \Tend{i}[}(t)$ est la fonction temps-caractéristique du 
traitement~$\treat_{\rm i}$ et~$\mu_{\rm i}$ représente le taux de mortalité\footnote{Le taux de mortalité dû au traitement est clairement lié à la dose de médicament délivrée au patient, mais pas seulement. Par exemple la sensibilité du patient et la dose parvenant vraiment à la tumeur sont aussi impliquées.} causé par~$\treat_{\rm i}$ sur les cellules proliférantes, pour~$\rm i\in\{1,2\}$. 
Le terme~$\vit (t,\vecx)$ désigne la vitesse impliquée dans les changements d'aire de la tumeur et~$M(t,\vecx)$ représente la vascularisation et l'apport en nutriments. 
Le terme $1+M$ traduit le fait que les traitements arrivent principalement par le réseau sanguin. 
Le taux de prolifération (respectivement de mortalité) des cellules proliférantes, noté~$\gammapp$ (respectivement~$\gammapd$), dépend de~$M$ comme suit~:
\begin{align}
\gammapp(M) &=\gamma_0 \frac{1+\tanh\big(R(M-\Ms)\big)}{2}, \label{eq_gammapp}\\
\gammapd(M) &=\gamma_1 \frac{1-\tanh\big(R(M-\Ms)\big)}{2}, \label{eq_gammapd}
\end{align}
où~$R$ est un paramètre numérique régularisant\footnote{Notons que les fonctions~$\gammapp$ et~$\gammapd$ sont des fonctions de Heaviside régularisées. Pour les simulations numériques, nous fixerons arbitrairement~$R$ à 5.}, $\gamma_0$ et
$\gamma_1$ sont respectivement les paramètres de 
prolifération/mortalité et~$\Ms$ est le seuil d'hypoxie. 
Cette forme des coefficients assure que lorsque~$M$ est au-dessus du seuil d'hypoxie~$\Ms$ alors~$\gammapp$ est supérieur à sa valeur moyenne~$\gamma_0/2$ et~$\gammapd$ est inférieur à sa valeur moyenne~$\gamma_1/2$. De plus, si~$R$ est grand, $\gammapp$ est proche de~$\gamma_0$ et~$\gammapd$ est proche de~0 dès lors que~$M>\Ms$ et de façon symétrique, si~$M<\Ms$ alors~$\gammapp$ est proche de~0  et~$\gammapd$ est proche de~$\gamma1$.



On suppose que les cellules saines sont uniquement sensibles à l'hypoxie et qu'elles sont passivement transportées par les changements de l'aire tumorale~:
\begin{equation}
\partial_t S + \dive(\vit S) =  -\gammasd(M) S, \label{eq_S}
\end{equation}
où~$\gammasd$ est le taux de mortalité des cellules saines dû à l'hypoxie~:
\begin{equation}
\gammasd(M)=C_S \gamma_1  \max\Big(0,-\tanh\big(R(M-\Ms)\big)\Big). \label{eq_gammasd}
\end{equation}
La fonction~$\gammasd$  s'annule exactement si~$M\geq\Ms$ dans le but d'assurer que~$S=1$ à l'extérieur de la lésion en tout temps. Finalement, la densité de tissus nécrosés satisfait la relation 
\begin{equation}\label{eq_N}
\begin{aligned}
\partial_t N + \dive(\vit N) & =  \gammapd(M) P +\gammasd(M) S  + \\ & \big(\muI \chiI(t)P_1 + \muS\chiS(t)(P_1+P_2)\big)(1+M) - \muN(1+M)N,
\end{aligned}\end{equation}
où
\begin{equation}
P = P_1+P_2+P_3,
\end{equation}
et où~$\muN$ est un paramètre qui contrôle le taux d'élimination de la nécrose par le système immunitaire. 
Le membre de droite de l'équation~\eqref{eq_N} collecte tous les termes de mort cellulaire des équations~\eqref{eq_P1},\eqref{eq_P2},\eqref{eq_P3} et \eqref{eq_S}. 
Le terme $1+M$ traduit le fait que plus le milieu est vascularisé, plus l'élimination de la nécrose est rapide. 
%Notez que b
%Biologiquement il ne serait pas aberrant de considérer que cette élimination pourrait dépendre également de la pression $\Pi$ du milieu (que l'on décrit un peu plus loin). 
%Ce type de considération conduit cependant à un modèle mathématiquement mal posé. 


La condition suivante de Dirichlet est utilisée sur le bord si la vitesse est entrante~:
\begin{equation}\label{eq_CL_pop}
P_1=P_2=P_3=N=1-S=0  \quad \textrm{ pour } \vecx \in \partial \Omega, \textrm{ si } \vit.\nn<0,
\end{equation}
où~$\nn$ est le vecteur normal sortant au domaine~$\Omega$.
%%%%%%%%%%%%%%%%%%%%%%%%%%%%%%%%%%%%%%%%%%%%%%%%
\subsubsection{Fonctionnement}

En utilisant la condition de saturation suivante 
(comme dans~\cite{ambrosi2002})
\begin{equation}\label{eq_saturation}
P+N+S=1,
\end{equation}
et en sommant~\eqref{eq_P1},\eqref{eq_P2},\eqref{eq_P3},\eqref{eq_S} et~\eqref{eq_N},
on obtient 
\begin{equation}\label{eq_divv}
\dive \vit = \gammapp P - \muN (1+M)N.
\end{equation}
La loi de Darcy assure la solvabilité du système, de manière similaire à~\cite{Ribba2006532}:
\begin{equation}\label{eq_darcy}
\left\{
\begin{aligned}
\vit(t,\vecx) &=-k\nabla \Pi(t,\vecx) & \quad \textrm{ dans }  \Omega , \\
\Pi(t,\vecx) &=0 & \quad \textrm{ sur  } \partial \Omega,
\end{aligned}
\right.
\end{equation}
où~$\Pi$ est la pression %(ou le potentiel) du 
dans le milieu et~$k$ sa perméabilité. 
La condition de Dirichlet homogène, dans~\eqref{eq_darcy}, impose qu'en dehors de~$\Omega$, la pression du tissu sain ne soit pas modifiée par la pression générée par la croissance tumorale. Cette hypothèse est valide pour de petites métastases n'impactant pas la forme du foie. En d'autres termes, la tumeur n'est pas mécaniquement contrainte par la région extratumorale. 


%Observez qu'une condition de flux nul impliquerait nécessairement que l'intégrale~$\int_{\Omega} \gammapp P - \delta (1+M)N \textrm{\;d}\vecx$ s'annule en tout temps, ce qui n'est pas satisfait à~$t=0$.


Nous soulignons que nous traitons ici de métastases hépatiques de GIST et non de tumeurs primaires du foie. Par conséquent, les cellules métastatiques ne grandissent pas dans leur milieu d'origine~: elles ne peuvent pas infiltrer le foie (comme le ferait un glioblastome dans le cerveau par exemple~\cite{Fathallah2013}) mais leur prolifération induit une pression qui pousse les cellules saines vers l'extérieur. Pour ces raisons, nous choisissons une loi de Darcy pour décrire la croissance de la métastase. 


Bien sûr, une loi de Stokes ou n'importe quelles autre loi plus complexe (loi viscoélastique par exemple) pourrait être utilisée, nous en référons à~\cite{Bresch2009}  et~\cite{ambrosi2002,byrne2003} pour plus de détails. 
De telles lois plus sophistiquées %compliquées 
peuvent fournir une meilleure description du front tumoral. Cependant, notre objectif ici n'est pas de fournir précisément le front tumoral ni la forme de la tumeur mais de reproduire la structure interne de la métastase.


%%%%%%%%%%%%%%%%%%%%%%%%%%%%%%%%%%%%%%%%%%%%%%%%
\subsubsection{Vascularisation, Nutriments et Angiogenèse}
Il reste à décrire l'apport de vascularisation/nutriments~$M$ 
et l'impact du traitement~$\TS$ sur celui-ci. Il est important de rappeler que la seconde ligne de traitement n'impacte pas directement~$M$, mais qu'elle bloque la production de facteurs de croissance qui gouvernent la quantité~$M$.

Nous introduisons une variable scalaire~$\xi$, qui est associée à la moyenne de facteurs de croissance produits par les cellules cancéreuses. 
Il a été signalé dans~\cite{Holmes2007} que les cellules hypoxiques augmentent leur production de facteurs de croissance, tandis que les cellules hautement proliférantes n'ont pas besoin d'apports nutritifs supplémentaires. Ainsi, si~$M$ est en dessous du seuil~$\Ms$ alors~$\xi$ doit augmenter. 
%Notez aussi que l
L'effet antiangiogénique du 
traitement~$\TS$ diminue la production
de~$\xi$, mais seulement pour les cellules~$P_1$ et~$P_2$ alors que~$P_3$ est la densité de cellules insensibles\footnote{Ceci est dû au fait que plusieurs voies moléculaires permettent d'engendrer la vascularisation. Le VEGF est le facteur de croissance vasculaire le plus commun  et est inhibé par le second traitement~$\TS$. La population $P_3$ émet un autre type de molécule que le VEGF pour se néovasculariser~: elle est donc insensible au traitement~$\TS$.} à~$\TI$ et~$\TS$.
Nous décrivons ainsi l'évolution de~$\xi$ par
\begin{equation}\label{eq_xi}
\partial_t \xi = \alpha \int_\Omega \big(1+\epsilon_{\xi} - \gammapp(M) / \gamma_0 \big)\Big(\big(1-\nuS\chiS(t)\big)(P_1+P_2)+P_3\Big) \textrm{\;d}\vecx - \lambda \xi.
\end{equation}
Le paramètre adimensionné~$\nuS\in [0;1]$ représente l'effet antiangiogénique de~$\TS$, qui est supposé identique pour~$P_1$ et~$P_2$, et le paramètre~$\epsilon_{\xi}$ traduit la production minimale de facteurs de croissance par les cellules cancéreuses. 


La quantité~$M$ décrit à la fois la vascularisation et la concentration de nutriments. Nous utilisons une équation de transport-réaction-diffusion pour décrire cette quantité~:
\begin{equation}\label{eq_M}
\left\{
\begin{aligned}
\partial_t M - \xi \frac{\nabla S}{\| \nabla S \|}\nabla M &= C_0 S
\left(1-\frac{M}{2\Ms}\right) - \eta PM +\psi \Delta M & \textrm{ dans } \Omega, \\
M(t,\vecx)&= 2\Ms & \textrm{ sur } \partial \Omega,
\end{aligned}
\right.
\end{equation}
où~$C_0$ est la capacité angiogénique des cellules saines, 
$\eta$ représente la destruction de la vascularisation par les cellules proliférantes et~$\psi$ est un paramètre de diffusion.  
Le terme de transport~$\xi \frac{\nabla S}{\| \nabla S \|}. \nabla M$ de~\eqref{eq_M} décrit le fait que la vascularisation est apportée à la tumeur depuis le tissu sain qui est supposé bien vascularisé.  
L'amplitude de la vitesse est égale à~$\xi$, qui représente le total des facteurs de croissance produits par les cellules proliférantes. 
Le terme de diffusion décrit l'infiltration des vaisseaux sanguins et des nutriments dans la tumeur. D'un point de vue numérique, ce terme a un effet régularisant sur la vascularisation~$M$ et ainsi stabilise le modèle. 
Le terme~$C_0S(1-\frac{M}{2\Ms})$ a deux rôles~:
\begin{myitemize}
\item C'est un terme de saturation pour la vascularisation qui empêche $M$ de croître indéfiniment. 
%Notez
Il est clair que si initialement 
\begin{align}\label{eq:borneM}
0\leq \|M|_{t=0}\|_{L^\infty}\leq 2\Ms
\end{align}
alors pour tout temps~$t$, $0\leq M(t) \leq 2\Ms$. 
\item Il permet d'imposer $M=2\Ms$ dans le tissu sain ($P=0$ et $S=1$) qui est bien vascularisé et est fourni de suffisamment de nutriments.
%Ceci traduit le fait que le tissu sain environnant est bien vascularisé et est fourni de suffisamment de nutriments.. 
\end{myitemize}

\bigskip
La quantité~$\xi$ est scalaire et représente la concentration totale de facteurs de croissances %(VEGF inhibé par le second traitement et autres facteurs de croissance ) 
produite par la tumeur. Cette description est bien entendu réductrice et il existe des modèles avec une répartition spatiale des facteurs de croissances (\cf~\cite{Billy2009545}) qui inclus leur production, leur diffusion, leur élimination, etc. Ces modèles sont très riches d'un point de vue biologique mais comprennent un grand nombre de paramètres qui serait difficile à gérer dans notre cas. 
%La quantité $\xi$ a été choisie scalaire. Des essais ont été réalisés en considérant~$\xi$ dépendant également de l'espace. L'équation régissant~$\xi$ était alors la même que l'équation~\eqref{eq_xi} mais sans l'intégrale. Cet essai a soulevé beaucoup de problème. Le principal, étant que~$\xi$, qui est produit par les cellules proliférantes, est concentré sur son lieu de production, \ie dans la tumeur, alors que la vascularisation~$M$ est concentrée à l'extérieur. Les variables~$\xi$ et~$M$ ont donc des supports quasiment disjoints, ce qui pose problème pour le transport de~$M$. Appliquer un transport et/ou une diffusion sur~$\xi$ aurait pu résoudre ce problème mais aurait complexifié le modèle qui compte déjà beaucoup de paramètres.

\subsection{Formulation vectorielle des équations sur les cellules cancéreuses}

Pour simplifier les notations, considérons le vecteur des populations de cellules~$\W=\trans{(P_1, P_2, P_3, N)}$. Soit la fonction vectorielle 
%$$
%\medmuskip=1mu
%\GG(M,\W) = \begin{pmatrix}
%\Big(\gammapp(M)- \gammapd(M) - \big(\muI \chiI (t)+\muS \chiS(t) \big)(1+M)\Big) P_1 \\
%\Big(\gammapp(M)- \gammapd(M) - \muS \chiS(t) (1+M)\Big)P_2 \\
%\big(\gammapp(M) - \gammapd(M) \big)P_3 \\
%\gammapd(M) P +\gammasd(M) (1-P-N)  + \big( \muI \chiI P_1 + \muS\chiS (P_1+P_2) \big) (1+M) - \muN(1+M)N
%\end{pmatrix}.
%$$
\begin{equation}
\GG(M,\W) = \trans{\big(\GG_1(M,\W), \GG_2(M,\W), \GG_3(M,\W), \GG_4(M,\W) \big)},
\end{equation}
dont les composantes sont données par~:
\begin{align}
\GG_1(M,\W) &= \Big(\gammapp(M)- \gammapd(M) - \big(\muI \chiI (t)+\muS \chiS(t) \big)(1+M)\Big) P_1, \\
\GG_2(M,\W) &= \big(\gammapp(M)- \gammapd(M) - \muS \chiS(t) (1+M)\big)P_2, \\
\GG_3(M,\W) &= \big(\gammapp(M) - \gammapd(M) \big)P_3, \\
\GG_4(M,\W) &= \gammapd(M) P +\gammasd(M) (1-P-N)  \nonumber \\
&\qquad +\big( \muI \chiI P_1 + \muS\chiS (P_1+P_2) \big) (1+M) - \muN(1+M)N.
\end{align}
Soit également~$\bar{\GG}$ la partie diagonale de l'opérateur~$\GG$~:
\begin{equation}\label{eq:GG_bar}
\bar{\GG} = \begin{pmatrix}
\gammapp(M)- \gammapd(M) - (\muI \chiI +\muS \chiS )(1+M) \\
\gammapp(M)- \gammapd(M) - \muS \chiS (1+M) \\
\gammapp(M) - \gammapd(M) \\
- \muN(1+M)
\end{pmatrix}.
\end{equation}
Il est également utile de définir~$\Frm(M,\W)$ comme
\begin{align}
  \label{eq_F}
  \Frm(M,\W):=\gammapp(M)\sum_{i=1}^3W_i-\muN(1+M)W_4,
\end{align}
de manière à réécrire l'ensemble d'équations~\eqref{eq_P1}-\eqref{eq_P3}, \eqref{eq_N} et~\eqref{eq_divv} fermées par la loi de Darcy de la manière suivante~:
\begin{equation}\label{eq_transport_pop}
\left\{
\begin{aligned}
\partial_t \W + (\nabla \W).\vit + \W (\dive \vit) &= \GG(M,\W) & \quad \textrm{ sur } & \Omega, \\
\W &=0 & \quad \textrm{ sur } & \partial \Omega \textrm{, si } \vit.\ \nn<0 ,
\end{aligned}
\right.
\end{equation}
et
\begin{equation}
\left\{ \begin{aligned}
 \label{eq_pression}
 -\dive\left(k\nabla \Pi\right) &=\Frm(M,\W), &\quad \textrm{ dans } & \Omega , \\
    \Pi(t,\vecx) &=0 & \quad \textrm{ sur  }& \partial \Omega.
\end{aligned}
\right. \end{equation}
Notons que la densité de cellules saines~$S$ est donnée par~\eqref{eq_saturation}.


%%%%%% Rajout par rapport à l'article
\subsection{Propriétés mathématiques du modèle}
\subsubsection{Positivité}
%\begin{prop}\label{prop:positivite_modele}
%Dans le modèle~\eqref{eq_transport_pop}, si
% $P_i(t,\vecx), N(t,\vecx) $ et $S(t,\vecx)$ appartiennent à~$L^\infty, \ \forall t \geq 0$, 
%alors $ \forall t\geq 0, \ P_i \geq 0, N \geq 0$ et $S\geq 0.$
%\end{prop}
\begin{prop}\label{prop:positivite_modele}
%%Dans le modèle~\eqref{eq_transport_pop}, si
Soient $P_i(t,\vecx), N(t,\vecx) $ et $S(t,\vecx)$ une solution de classe~$\mathscr{C}([0,T]\times \Omega)$ du système~\eqref{eq_transport_pop}. Supposons que la vitesse~$\vit$ est régulière et que~$M$ est bornée. Si $P_i(0,\vecx), N(0,\vecx)$ et~$S(0,\vecx)$ sont positives ou nulles alors 
$$\forall t \in [0,T], \quad  N(t,\vecx)\geq0, \quad  P_i(t,\vecx)\geq0, \quad S(t,\vecx)\geq0. $$
\end{prop}
Pour démontrer cette positivité, nous allons montrer que les parties négatives de ces quantités sont toutes nulles. Les parties négatives sont définies comme suit~:
\begin{dfn}[Parties positive et négative]~
\begin{itemize}
\item On appelle la \emph{partie positive} de~$f$, la quantité~$f^+=\max(0,f)$.
\item On appelle la \emph{partie négative} de~$f$, la quantité~$f^-=-\min(0,f)$.
\end{itemize}
\end{dfn}
On notera les propriétés intéressantes de ces quantités~:
\begin{prop}\label{prop:partie_neg_et_pos}
Quelle que soit la fonction~$f$, on a~:
\begin{itemize}
\item $f^+ \geq 0$ et $f^- \geq 0$
\item $ff^+=(f^+)^2$  et $ff^-=-(f^-)^2$ 
\item  $f'f^+=(f^+)'f^+$  et $f'f^-=-(f^-)'f^-$ 
\end{itemize} 
\end{prop}
%La première partie de cette propriété découle directement de la définition, la seconde se démontre de manière immédiate par disjonction des cas~$f\geq0$ et~$f<0$. Ayant définit ces parties négatives et positives, nous pouvons maintenant démontrer la positivité de notre modèle.

\begin{proof}[Démonstration de la Propriété~\ref{prop:positivite_modele}]
On suppose que $P_i,N$ et $S$ sont bornées. Commençons par démontrer la positivité de~$P_i$. On a~:
$$\partial_t P_i + \dive (\vit P_i) = \bar{\GG_i}P_i $$
En multipliant par la partie négative~$P_i^-$ puis en intégrant, on obtient l'estimation d'énergie suivante~:
$$
\begin{aligned}
&\qquad \intx (\partial_t P_i) P_i^- + \intx \dive (\vit P_i) P_i^- = \intx \bar{\GG_i} P_i P_i^- \\
&\overset{IPP\protect\footnotemark}{\Longleftrightarrow} \intx (\partial_t P_i) P_i^- + \intbord \vit . \nn P_i P_i^- - \intx (\vit P_i).\nabla (P_i^-)  = \intx \bar{\GG_i} P_i P_i^-  \\
&\overset{\textrm{Prop. \ref{prop:partie_neg_et_pos}}}{\Longleftrightarrow}  - \intx (\partial_t P_i^-) P_i^- - \intbord \vit . \nn (P_i^-)^2 + \intx P_i^- \vit.\nabla (P_i^-)  = - \intx \bar{\GG_i} (P_i^-)^2  \\
&\Longleftrightarrow  \frac{-1}{2} \intx \partial_t \big( (P_i^-)^2 \big) - \intbord \vit . \nn (P_i^-)^2 + \frac{1}{2} \intx \vit.\nabla \big( (P_i^-)^2\big)   = - \intx \bar{\GG_i} (P_i^-)^2  \\
&\overset{IPP}{\Longleftrightarrow}  \frac{-1}{2} \intx \partial_t \big( (P_i^-)^2 \big) -\frac{1}{2} \intbord \vit . \nn (P_i^-)^2 - \frac{1}{2} \intx (\dive \vit)  (P_i^-)^2  = - \intx \bar{\GG_i} (P_i^-)^2  \\
%%&\Longleftrightarrow  \intx \partial_t \big( (P_i^-)^2 \big) + \intbord \vit . \nn (P_i^-)^2 + \intx (\dive \vit)  (P_i^-)^2  = 2 \intx \bar{\GG_i} (P_i^-)^2  \\
\end{aligned}$$
\footnotetext{IPP~: Intégration Par Parties.}
Ainsi
\begin{equation}
\partial_t \intx (P_i^-)^2  = \intx (2\bar{\GG_i} - \dive \vit)  (P_i^-)^2 - \intbord \vit . \nn (P_i^-)^2.
\end{equation}
Or $$\intbord \vit . \nn (P_i^-)^2 \geq 0. $$
Si~$\vit.\nn\geq0$, ceci est immédiat, sinon la condition limite impose~$P_i=0$. 
Les quantités~$P_i$ , $N$ et~$M$ étant bornées, alors~$\dive \vit$ et~$\bar{\GG}_i$ sont bornés. Ceci nous conduit à la majoration suivante~:
\begin{equation}\label{eq:demo_positivite}
\partial_t \intx (P_i^-)^2 \leq \| 2\bar{\GG_i} - \dive \vit \|_{\infty} \intx (P_i^-)^2.
\end{equation}
En posant $y(t):=\displaystyle \intx (P_i^-(t))^2$, le lemme de Gronwall nous conduit à
\begin{equation}
y(t) \leq y(t=0) \exp ( \| 2\bar{\GG_i} - \dive \vit \|_{\infty} t  ).
\end{equation}
Or~$P_i\geq 0$ au temps initial, donc~$y(t=0)=0$. Il en découle que~$y(t)\leq0, \forall t\geq0$. Mais par définition, $y$ est nécessairement positive. Ainsi~$y$ est identiquement nulle. Ceci conduit à~$P_i^-(t)\equiv0, \forall t\geq0$ et donc $P_i(t)\geq0, \forall t\geq0$. \\
Le même type de raisonnement permet de montrer la positivité de~$S$ (il suffit de remplacer~$P_i$ par~$S$ et~$\bar{\GG}_i$ par~$-\gammasd$ qui est borné). Pour la positivité de~$N$, sachant que~$P$ et~$S$ le sont alors on a
$$ \partial_t N + \dive (\vit N) \geq \bar{\GG}_4 N.$$
En procédant de la même manière et en remplaçant les égalités par la minoration ci-dessus, on obtient également la majoration~\eqref{eq:demo_positivite} avec~$N^-$, de laquelle découle la positivité de~$N$ en tout temps positif. Ceci clôt la démonstration.
\end{proof}


\subsubsection{Préservation de la densité totale}
Dans ce paragraphe, on considère~$\W=\trans{(P_1, P_2, P_3, N, S)}$ dans sa forme pleine \ie avec~$S$. On pose~:
\begin{equation}\label{eq:G5}
\GG_5(M,\W) = -\gammasd(M) S,
\end{equation}
pour compléter la formulation~\eqref{eq_transport_pop}.
\begin{prop}
%Le vecteur~$\W=\trans{(P_1, P_2, P_3, N, S)}$ vérifie une condition de saturation \ssi la somme d'un transport conservatif sur chacune de ses composantes est égale à la divergence de la vitesse régissant ce transport.
%\begin{equation}
%\sum_i \W_i = 1 \quad \Leftrightarrow \quad \sum_i  [ \partial_t \W_i + \dive(\vit\W_i) ] = \dive \vit
%\end{equation}
Si $(P+N+S)(t=0,\vecx)=1,\ \forall \vecx \in \Omega$ alors 
\begin{equation}
(P+N+S)(t,\vecx)=1,\quad \forall \vecx \in \Omega, \quad \forall t\geq 0.
\end{equation}
\end{prop}
\begin{proof}
Pour réaliser cette preuve, notons $\sigma := \sum_i \W_i = P+N+S$. %\\
%$\Rightarrow$) Quasiment immédiat.
%\begin{align*}
%\sum_i  [ \partial_t \W_i + \dive(\vit\W_i) ] & = \sum_i  [ \partial_t \W_i + \dive(\vit) \W_i + \vit . \nabla \W_i ] \\
%& = \partial_t \sigma + (\dive\vit) \sigma + \vit . \nabla \sigma = \dive \vit,
%\end{align*}
%puisque les dérivées de $\sigma$  sont nulles car ici il est constant. \\
%$\Leftarrow$) 
%On suppose ici que 
En utilisant les équations~\eqref{eq_P1},\eqref{eq_P2},\eqref{eq_P3},\eqref{eq_S},\eqref{eq_N} et~\eqref{eq_divv}, un calcul immédiat montre que 
$\displaystyle\partial_t \sigma + \dive (\vit \sigma ) = \sum_i  [ \partial_t \W_i + \dive(\vit\W_i) ]  = \dive \vit$ ou autrement écrit que $$ \partial_t \sigma + \vit . \nabla \sigma  = (1-\sigma) \dive \vit.$$ Montrons que ceci implique que~$\sigma\equiv1$. Remarquons que $\sigma\equiv1$ est solution de l'équation ci-dessus. Il ne reste donc plus qu'à montrer son unicité pour conclure. Considérons~$\sigma_1$ et~$\sigma_2$, deux solutions de cette équation. Soit $d:=\sigma_2-\sigma_1$. On a alors~:
$$ \partial_t d + \dive (\vit d) = 0 \textrm{~~avec~~} d(t=0,\vecx)\equiv0. $$
Par le même procédé qu'utilisé précédemment (estimation d'énergie pour démontrer la positivité de la solution), on a~:
\begin{equation}
\partial_t \intx d^2 + \intx d^2 \dive \vit = - \intbord (\vit . \nn) d^2 \leq 0.
\end{equation}
Ainsi par le lemme de Gronwall, appliqué à la fonction~$\left(\intx d^2 \right)(t)$, on a~:
\begin{equation}
\left(\intx d^2 \right)(t) \leq \left(\intx d^2 \right)(t=0) \exp ( \| \dive \vit \|_{\infty} t ) = 0
\end{equation}
Donc~$d(t,\vecx)^2\equiv0, \forall t$, donc $d$ est identiquement nulle. Ceci conduit donc à~$\sigma_1=\sigma_2$, d'où l'unicité de la solution. 
Il en découle donc $P+N+S=1, \forall t\geq 0$.
\end{proof}

\subsubsection{Comportement asymptotique lorsque le taux d'élimination de la nécrose tend vers l'infini ($\delta \to +\infty$)}
Nous allons montrer ici que notre modèle tend, lorsque le taux d'élimination devient grand, vers un modèle sans nécrose. Nous allons procéder sur un modèle un peu plus simple. Les 3 populations $P_1,P_2$ et~$P_3$ vérifiant des équations similaires, nous nous restreindrons à une seule population de cellules proliférantes. Le raisonnement étant analogue en considérant l'effet cytotoxique des traitements ou non, nous ne considèrerons pas cette aspect pour simplifier les calculs. D'autre part, nous allons ici considérer la vascularisation~$M$ constante (dans l'espace et le temps). Posons 
\begin{equation}
\epsilon:= \frac{1}{\muN(1+M)}.
\end{equation}
On a alors la propriété suivante~:
\begin{prop} Le modèle~A avec nécrose 
\begin{subequations}\label{eq:modele_simplifie1}
\begin{empheq}[left={(\rm A)\quad}\empheqlbrace]{align} %%%% grace au package empheq
\partial_t P + \nabla.(vP)& = ( \gammapp - \gammapd  ) P, \label{modele1-eq1}\\
\partial_t N + \nabla.(vN)& =\gammapd P- N/\epsilon, \label{modele1-eq2} \\
\nabla.v & = \gammapp P- N/\epsilon,
\end{empheq}
\end{subequations}
converge vers le modèle~B sans nécrose
\begin{subequations}\label{eq:modele_sans_necrose}
\begin{empheq}[left={(\rm B)\quad}\empheqlbrace]{align} %%%% grace au package empheq
\partial_t P + \nabla.(vP) & = ( \gammapp - \gammapd  ) P, \\
\nabla.v & = ( \gammapp - \gammapd  ) P.
\end{empheq}
\end{subequations}
lorsque $\epsilon$ tend vers 0.
\end{prop}

%et montrons que lorsque~$\epsilon \to 0$, le modèle~A avec nécrose 
%\begin{subequations}\label{eq:modele_simplifie1}
%\begin{empheq}[left={(\rm A)\quad}\empheqlbrace]{align} %%%% grace au package empheq
%\partial_t P + \nabla.(vP)& = ( \gammapp - \gammapd  ) P, \label{modele1-eq1}\\
%\partial_t N + \nabla.(vN)& =\gammapd P- N/\epsilon, \label{modele1-eq2} \\
%\nabla.v & = \gammapp P- N/\epsilon,
%\end{empheq}
%\end{subequations}
%converge vers le modèle~B sans nécrose
%\begin{subequations}\label{eq:modele_sans_necrose}
%\begin{empheq}[left={(\rm B)\quad}\empheqlbrace]{align} %%%% grace au package empheq
%\partial_t P + \nabla.(vP) & = ( \gammapp - \gammapd  ) P, \\
%\nabla.v & = ( \gammapp - \gammapd  ) P.
%\end{empheq}
%\end{subequations}

\begin{proof}Considérons une nouvelle variable~$Z$ définie comme suit 
\begin{equation}
 Z=\gammapd P- N/\epsilon.
\end{equation}
La combinaison~$\gammapd $ \eqref{modele1-eq1}-\eqref{modele1-eq2}$/\epsilon$ nous donne que le modèle~A est équivalent à~:
\hspace*{-35mm}\begin{subequations}\label{eq:modele_simplifie1}
\begin{empheq}[left={(\rm A) \Leftrightarrow \ }\empheqlbrace]{align} %%%% grace au package empheq
\partial_t P + \nabla.(vP) & = ( \gammapp - \gammapd  ) P \label{modele1equiv-eq1}\\
\partial_t Z + \nabla.(vZ) & = -Z/\epsilon + r P  & \textrm{avec} \quad r=\gammapd ( \gammapp - \gammapd  ) \label{modele1equiv-eq2} \\
\nabla.v & =  Z
\end{empheq}
\end{subequations}
En multipliant \eqref{modele1equiv-eq2} par $Z$, puis en intégrant en espace, on obtient :
\begin{equation}
\intx Z \partial_t Z + \intx Z \nabla.(vZ) = - \frac{1}{\epsilon} \intx Z^2 + \intx rPZ
\end{equation}
Or une série de deux intégrations par parties (IPP) nous fournit l'égalité suivante~:
\begin{align*}
\intx Z \partial_t Z + \intx Z \nabla.(vZ) & \overset{IPP}{=} \intx Z \partial_t Z - \intx v Z . \nabla Z + \intbord (v.\vec{n}) Z^2 \\
&= \frac{1}{2} \partial_t \intx Z^2 - \frac{1}{2} \intx v . \nabla (Z^2) + \intbord (v.\vec{n}) Z^2 \\
& \overset{IPP}{=} \frac{1}{2} \partial_t \intx Z^2 + \frac{1}{2} \intx Z^2 \nabla.v + \frac{1}{2} \intbord (v.\vec{n}) Z^2 \\
\end{align*}
D'où :
\begin{equation}
\partial_t \intx Z^2 + \intx Z^2 \nabla.v = - \frac{2}{\epsilon} \intx Z^2 + 2 \intx rPZ - \intbord (v.\vec{n}) Z^2
\end{equation}
Or
\begin{equation}
\intbord (v.\vec{n}) Z^2 \geq 0.
\end{equation}
En effet si $v.\vec{n} \geq 0$ c'est trivial, sinon les C.L ($P=N=0$ sur $\partial \Omega$) impliquent que $Z=0$ sur $\partial \Omega$. Dans ce dernier cas, le terme est donc nul. Ainsi :
\begin{equation}
 \partial_t \intx Z^2 \leq - \frac{2}{\epsilon} \intx Z^2 + 2 \intx rPZ - \intx Z^2 \nabla.v
\end{equation}
Considérons une solution suffisamment régulière de notre problème de telle sorte que $\nabla.v \in L^\infty$, $P \in L^2, N \in L^2$ et par conséquent $Z \in L^2$. De plus, $r$ est constant (\ie ne dépend ni de l'espace ni du temps) puisque nous avons considéré~$M$ constant ici. Par l'inégalité triangulaire et l'inégalité de Young\footnote{L'inégalité de Young affirme que pour tout couple de réels $(a,b)$ positifs et pour tout couple de réels $(p,q)$ positifs et conjugués (\ie tel que $\frac{1}{p}+\frac{1}{q}=1$) alors on a~: $ab \leq \frac{a^p}{p} + \frac{b^q}{q}$. L'inégalité est utilisée ici avec $p=q=2$. }, 
on obtient alors~:
\begin{align*}
\partial_t \| Z \|^2_2 & \leq -\frac{2}{\epsilon} \intx Z^2 + 2 \intx  |r| \dfrac{\sqrt{\epsilon}}{\sqrt{\epsilon}} P|Z| + \intx Z^2 | \nabla.v | \\
& \leq -\frac{2}{\epsilon} \intx Z^2 + \intx \epsilon r^2 P^2 + \intx \dfrac{Z^2}{\epsilon}  + \intx Z^2 | \nabla.v | \\
& \leq \intx \epsilon r^2 P^2 - \dfrac{1}{\epsilon} \intx Z^2  + \intx Z^2 | \nabla.v | 
\end{align*}
Ainsi :
\begin{equation}
\partial_t \|Z\|_2^2 \leq \epsilon r^2 \|P\|_2^2 + \left( \|\nabla.v\|_\infty - \frac{1}{\epsilon} \right) \| Z \|^2_2
\end{equation}
Par le lemme de Gronwall, on a $\forall t\in [0,T]$ :
\begin{equation}
\begin{array}{l}
\displaystyle \|Z(t)\|^2_2 \leq \|Z(t=0)\|_2^2 \exp \left(  \left(  |\nabla.v\|_\infty - \frac{1}{\epsilon} \right) t\right) \\
\displaystyle \quad + \epsilon r^2 \max_{t\in[0,T]} \| P(t) \|^2_2 \int_0^t \exp \left(  \left(  |\nabla.v\|_\infty - \frac{1}{\epsilon} \right) (t-s) \right) \textrm{d}s
\end{array}
\end{equation}

Lorsque $\epsilon \to 0$, on obtient que $Z \to 0$. Ainsi, par définition de~$Z$, $N$ tend vers 0 à la même vitesse que~$\epsilon$. Ceci montre que le modèle A avec nécrose converge vers le modèle sans nécrose lorsque $\epsilon \to 0$. 
\end{proof}

Pour ce qui est du modèle complet, des simulations numériques ont été réalisées avec de grandes valeurs de~$\muN$. On constate bien une convergence vers~$N\equiv0$. 
Le comportement asymptotique du modèle complet (incluant le couplage avec la vascularisation~$M$) a été étudié également de manière théorique et plus précise  
%Le résultat a été également étudié sur le système complet 
dans~\cite{michel:hal-01211770}. Ce type de résultat montre que notre système est uniformément valide en fonction du paramètre~$\muN$, \cad qu'il est capable de décrire de façon continue des situations où la nécrose s'évacue instantanément ou bien avec un temps caractéristique. Ceci est important dans notre cas, car il est difficile d'estimer ce temps caractéristique.

\section{Méthodes numériques}\label{sec:NumMet}

Nous utilisons une méthode de volumes finis sur grille uniforme 2D~\cite{eymard2000finite}. 
Nous avons choisi de travailler en 2D pour coller aux habitudes cliniques des radiologistes. En effet, pour l'évaluation de la réponse à un traitement, le critère RECIST impose de mesurer le diamètre de la métastase sur la coupe sur laquelle il est le plus grand. L'évaluation est donc réalisée dans un cadre 2D.


Pour la simulation numérique, le domaine~$\Omega$ est le 
rectangle~$[0,L]\times[0,D]$. Le domaine est maillé avec une grille cartésienne contenant~$N_x$ points sur l'axe~$x$ et~$N_y$ points sur l'axe~$y$, avec des mailles décalées pour la vitesse et les populations de cellules. 
Les densités de cellules sont discrétisées au centre des mailles et les vitesses sont discrétisées au milieu de chaque arête comme montrées sur la Figure~\ref{fig:place_inconnues}.

\begin{figure}[h]
\centering
\setlength{\unitlength}{1.4mm}
\begin{picture}(100,50)

\rect{20}{10}{30}{30} %% \rect est defini dans mes macros

\multiput(16,10)(0,15){3}{\multiput(0,0)(1.5,0){26}{\line(1,0){0.5}}}
\multiput(20,6)(15,0){3}{\multiput(0,0)(0,1.5){26}{\line(0,1){0.5}}}

\multiput(20,25)(30,0){2}{\oval(4,2)}
\multiput(35,10)(0,30){2}{\oval(2,4)}
\rect{33.5}{23.5}{3}{3}
\put(6,27){$(i-\frac{1}{2},j)$}
\put(37,21){$(i,j)$}
\put(51,21){$(i+\frac{1}{2},j)$}
\put(36.5,42){$(i,j+\frac{1}{2})$}
\put(36.5,6){$(i,j-\frac{1}{2})$}

\put(15,5){\vector(1,0){9}}
\put(15,5){\vector(0,1){9}}
\put(14,15){$\ey$}
\put(25,4){$\ex$}

%% legende
\rect{63.5}{39.5}{3}{3}
\put(65,36){\oval(4,2)}
\put(65,31){\oval(2,4)}
\put(69,40){$P_1,P_2,P_3,N,S,M$}
\put(69,35){$v_x$}
\put(69,31){$v_y$}

\end{picture}
\caption{Discrétisation des inconnues sur une maille.\label{fig:place_inconnues}}
\end{figure}


Remarquons que l'égalité~\eqref{eq_saturation} donne directement~$S$, sans résoudre l'équation~\eqref{eq_S}. Le calcul est séparé en plusieurs étapes, comme suit~:
\begin{myitemize}
\item Etant donnés\footnote{L'exposant~$^n$ représente le temps discret~$t^n$ de la quantité 
(par exemple~$S^n$ est la densité de cellules saines au temps~$t^n$)}
$\W^n$ et~$M^n$ au temps~$t^n$, on infère~$\Frm^n=\Frm(M^n,\W^n)$ et
$\GG^n=\GG(M^n,\W^n)$.
\item On calcule d'abord la pression~$\Pi^n$ solution de l'équation~\eqref{eq_pression} avec~$\Frm^n$ comme membre de droite, pression à partir de laquelle on en  déduit la vitesse~$\vit^n$ grâce à l'équation~\eqref{eq_darcy}.
\item Ensuite le nouveau temps~$t^{n+1}=t^n+\Delta t$ est déterminé en utilisant la condition  CFL~\eqref{eq:dt}.
\item Après cela, $\W^{n+1}$ est calculé à partir de~\eqref{eq_transport_pop}, à partir duquel on infère~$S^{n+1}$.
\item On termine par le calcul de~$\xi^{n+1}$ et de~$M^{n+1}$ grâce à~\eqref{eq_xi}--\eqref{eq_M}.
\end{myitemize}
Présentons maintenant dans le détail, les schémas utilisés dans les simulations numériques. 
Les aspects techniques concernant l'intégration de ces schémas dans un code de calcul C++, sont quant à eux fournis dans l'Annexe~\ref{chap:anx_img_complement}.

\subsection{Calcul de la vitesse et de la pression} 


Selon~\eqref{eq_divv} et~\eqref{eq_darcy}, 
la pression~$\Pi^n$ est donnée par
\begin{equation}\label{eq_pression_disc}
\left\{
\begin{aligned}
-\dive (k\nabla \Pi^n)&=\Frm^n:=\Frm(M^n,\W^n) & \quad \textrm{ sur } & \Omega , \\
\Pi^n &=0 & \quad \textrm{ sur } & \partial \Omega ,
\end{aligned}
\right.
\end{equation}
où~$\Frm$ est définie par~\eqref{eq_F}. 
Cette équation est résolue grâce à un schéma classique à 5 points. 


Une fois la pression $\Pi$ calculée, la vitesse est donnée par dérivation de~$\Pi$ (par la loi de Darcy~\eqref{eq_darcy}).
La composante de la vitesse selon~$\ex$, $v^{x,n}_{i+1/2 j}$ au point 
$\vecx_{i+1/2\,j}$ ({\it resp.} la composante selon~$\ey$, $v^{y,n}_{i,
  j+1/2}$ au point~${\vecx_{i\,j+1/2}}$) est donnée grâce à
\begin{equation*}
v^{x,n}_{i+1/2,j} = -k \frac{\Pi^n_{i+1,j}-\Pi^n_{i,j}}{\Delta x},\quad
v^{y,n}_{i,j+1/2} = -k \frac{\Pi^n_{i,j+1}-\Pi^n_{i,j}}{\Delta y}
\end{equation*}
et la vitesse~$\vit^n_{ij}$ au point~$\vecx_{ij}$ est approchée par
\begin{align}
  \label{eq_vit_discr}
  \vit^n_{ij}=    v^{x,n}_{ij}\ex
  +v^{y,n}_{ij}\ey=\frac{1}{2}\left(v^{x,n}_{i+1/2,j}
    +v^{x,n}_{i-1/2,j} \right) \ex+\frac{1}{2}\left(v^{y,n}_{i,j+1/2}
    +v^{y,n}_{i,j-1/2} \right) \ey.
\end{align}

\subsection{Equation d'advection}
Par la définition~\eqref{eq_F} de~$\Frm$, l'équation~\eqref{eq_divv} 
conduit à~$\dive\vit=\Frm$. 
Ainsi, l'équation~\eqref{eq_transport_pop} peut être écrite sous la forme non conservative
\begin{equation}\label{eq_non_conservative}
\partial_t \W + (\nabla \W).\vit = \Grm(M,\W)-(\dive \vit)\W = \Grm(M,\W)-\Frm(M,\W)\W,
\end{equation}
résolue grâce au spliting en temps suivant~:
\begin{align}
 \dfrac{\W^*-\W^n}{\Delta t / 2}  &= \GG^n-\Frm^n\W^n, %% & \textrm{then}
 \label{eq_snd_membre} \\
\dfrac{\W^\#-\W^*}{\Delta t} + \vit^n.\nabla \W^*&= 0,%%& \textrm{then} 
 \label{eq_transp_weno}\\
 \dfrac{\W^{n+1}-\W^\#}{\Delta t / 2}  &= \GG^n-\Frm^n\W^n. \label{eq_snd_membre2} 
\end{align}
Cette forme non conservative est choisie dans le but de réutiliser les informations que nous possédons déjà sur la divergence de la vitesse, grâce à la relation~\eqref{eq_divv}.


Une méthode de type WENO5 comme donnée par~\cite{shu1988efficient,Liu1994,gottlieb2006fifth} est utilisée pour approcher le gradient~$\nabla \W$ impliqué  dans~\eqref{eq_transp_weno}. Cette variante du WENO5, appelée \twinweno, est présentée dans le chapitre suivant (section~\ref{sec:twinweno}). 
D'autres schémas d'ordre moins élevé ont aussi été essayés. Cependant la diffusion numérique de ces schémas est trop importante. Ceci conduit à une diffusion artificielle des nutriments dans la tumeur. La croissance tumorale est alors modifiée par cet artefact. La résolution précise du transport est par conséquent un point important. De plus, l'utilisation d'un schéma d'ordre élevé réduit, par le gain de convergence, les problèmes usuels liés à la résolution du problème non conservatif.

\subsection{Calcul des quantités~$\xi$ et~$M$}
L'équation scalaire~\eqref{eq_xi} est résolue grâce à la méthode d'Euler explicite~:
\begin{equation}\label{eq:xi_disc}
\dfrac{\xi^{n+1}-\xi^n}{\Delta t} = \alpha \int_\Omega \big(1+\epsilon_{\xi} - \gammapp(M^n) / \gamma_0 \big)\Big((P_1^n +P_2^n) \big(1-\nuS \chiS(t^n) \big)+P_3^n \Big) \textrm{\;d}\vecx - \lambda \xi^n,
\end{equation} 
où la formule standard des rectangles est utilisée pour approcher l'intégrale du membre de droite.


Pour résoudre l'équation d'advection-diffusion~\eqref{eq_M}
satisfaite par~$M$, le spliting en temps suivant est réalisé~:
\begin{align}
\dfrac{M^*-M^n}{\Delta t/2} &= C_0 S^n \left( 1-\dfrac{M^n}{2\Ms}\right) - \eta P^nM^n, %%& \textrm{then} 
\\
\dfrac{M^{\#}-M^*}{\Delta t} - \psi \dfrac{ \Delta M^\# + \Delta M^*}{2} &= \xi^n  \dfrac{\nabla S^n}{\| \nabla S^n \| }\nabla M^n, %%& \textrm{then} 
\label{eq_transp_diff_M}\\
\dfrac{M^{n+1}-M^\#}{\Delta t/2} &= C_0 S^n \left( 1-\dfrac{M^n}{2\Ms}\right) - \eta P^nM^n. \label{eq_snd_membre_M}
\end{align}
L'équation~\eqref{eq_transp_diff_M} est calculée comme une équation de la chaleur un terme source.  Pour l'opérateur de diffusion, un schéma standard à 5 points de grille est utilisé. Le membre de droite est approché par un schéma \twinweno également.


%%%%%%%%%\subsection{Le \twinweno~: un schéma WENO5 modifié}
\subsection{Condition CFL}

En outre, une restriction de type CFL est requise pour assurer la stabilité numérique. Tout d'abord, le schéma de type WENO5, utilisé dans~\eqref{eq_transp_weno} et~\eqref{eq_transp_diff_M}, conduit à une condition CFL qui s'écrit~:
\begin{equation}\label{eq:CFL-weno}
\Delta t < \min \left( \frac{\Delta x}{\max |v_x|} , \frac{\Delta y}{\max |v_y|}, \frac{\min(\Delta x , \Delta y )}{\xi}\right) := \Delta t_{adv}.
\end{equation}
Le schéma d'Euler explicite sur l'équation~\eqref{eq_snd_membre} conduit à l'inégalité 
(coordonnées par coordonnées)
$$
\W^{*}=\W^n+\frac{\Delta t}{2} (\GG^n -\Frm^n\W^n) \geq 
\left( \mathbf{1}+\frac{\Delta t}{2} (\bar{\GG}^n - \Frm^n\mathbf{1} )\right) \circ \W^n,
$$
où~$\mathbf{1}=\trans{(1,1,1,1)}$, le symbole~$\circ$ représente le produit d'Hadamard (multiplication terme à terme) et où~$\bar{\GG}$ est défini par~\eqref{eq:GG_bar}. 
Une relation similaire entre~$\W^\#$ et~$\W^{n+1}$ 
peut être écrite depuis l'équation~\eqref{eq_snd_membre2}. 
Ainsi, en supposant que~$\W^n\geq 0$ et~$S^n=1-\Sigma_i W^n_i\geq 0$ 
au temps~$t^n$, la restriction suivante sur le pas de temps
\begin{equation}\label{eq:CFL-spliting-pop}
\Delta t < \min \left( \dfrac{1}{ \max_i \|  \bar{\GG}_i^n-\Frm^n \|_\infty} , \dfrac{1}{ \| \gammasd(M^n) + \Frm^n  \|_{\infty}} \right) := \Delta t_{W}.
\end{equation}
assure que~$\W^{n+1}\geq 0$ et~$S^{n+1}\geq 0$. 
De manière similaire, le schéma d'Euler explicite dans les équations~\eqref{eq:xi_disc} et~\eqref{eq_snd_membre_M} 
sur la vascularisation, conduit à
\begin{equation}\label{eq:CFL-spliting-vasc}
\Delta t < \min \left(  \frac{1}{\eta} , \frac{1}{\lambda}\right) := \Delta t_{angio}.
\end{equation}
Finalement, comme la vitesse peut être très petite, pour éviter les pas de temps 
$\Delta t$ trop grands, nous choisissons arbitrairement une vitesse~$\bar{\vit}$ et notre condition de stabilité globale s'écrit\footnote{Dans les simulations, nous choisissons 
$\bar{\vit}=1\;cm/mois$, en considérant~$30\;jours$ par mois et~$C_{\CFL}=0.4$.} 
\begin{equation}\label{eq:dt}
\Delta t =C_{\CFL}\min \left( \frac{1}{{\bar{\vit}}} \min( \Delta x, \Delta y) , \Delta t_{adv}, \Delta t_{W} , \Delta t_{angio} \right),
\end{equation}
pour une constante donnée~$C_{\CFL}<1$.

Un schéma implicite aurait pu être utilisé pour relaxer les conditions~\eqref{eq:CFL-spliting-pop}-\eqref{eq:CFL-spliting-vasc}. 
Cependant avec l'ensemble de paramètres utilisé, le pas de temps est principalement limité par~$\Delta t_{adv}$. Ainsi, procéder de manière implicite ne modifie pas la condition~\eqref{eq:dt}.

%%%%%%%%%%%%%%%%%%%%%%%%%%%%%%%%%%%%%%%%
\section{Résultats numériques}\label{sec:NumRes}

\subsection{Outils numériques pour comparer les résultats aux scanners}

Afin de comparer les résultats numériques aux scanners, nous devons définir des quantités d'intérêts appropriées et 
%aussi bien que 
développer un outil numérique qui reproduit l'échelle de gris. 
\subsubsection{Détermination numérique de l'aire tumorale, de la partie nécrosée et de la masse tumorale}

Soit le seuil~$\epsilon_{th}$, la fraction minimale de cellules tumorales en dessus de laquelle nous définissons numériquement la tumeur. 
L'aire tumorale, mesurée numériquement, s'écrit alors
\begin{equation}
\label{eq:airetot}
\mathcal{A}(t)=\int_\Omega \mathds{1}_{\{\vecx\;: P(t,\vecx)+N(t,\vecx) >\epsilon_{th}\}}(\vecx)\textrm{\;d}\vecx.
\end{equation}
Définissons l'aire de chaque population cancéreuse par
\begin{equation}
\mathcal{A}_J(t)=\int_\Omega
\mathds{1}_{\{\vecx\,:\,J(t,\vecx)>\epsilon_{th}\}}(\vecx)\textrm{\;d}\vecx,
\qquad \text{pour}\quad J \in \{P_1,P_2,P_3,N \}.
\label{eq:airepop}
\end{equation}
Définissons également la masse de chaque population et la masse proliférante totale en tout temps par 
\begin{align}
\mathcal{M}_J(t)&=\int_\Omega J(t,\vecx) \textrm{\;d}\vecx, \quad J \in \{P_1,P_2,P_3,N \},
\label{eq:masspop} \\
\mathcal{M}(t)&=\int_\Omega P 
(t,\vecx)\textrm{\;d}\vecx.
\label{eq:masstot}
\end{align}


\subsubsection{Reconstitution numérique de scanners}

La comparaison de simulations numériques 
avec les scanners soulève une difficulté importante. 
Actuellement, contrairement aux simulations numériques qui fournissent en tout point chacune des quantités~$S$, $P$ et~$N$, les scanners fournissent des images en niveaux de gris reliés à la radiodensité locale grâce à l'échelle de 
Hounsfield. Comme les unités Hounsfield (HU) rendent possible la quantification  
l'aire de la tumeur et la détection de sa localisation sur les scanners, nous introduisons une échelle numérique HU, qui est une combinaison linéaire de nos résultats numériques. 
Plus précisément, nous considérons une échelle linéaire de gris allant du noir au blanc. Pour chaque espèce (cellules proliférantes, saines et nécrosées), nous allouons les 
coefficients~$\tau_P$, $\tau_S$ et~$\tau_N$ à chacune de nos populations   et nous traçons ensuite la quantité
\begin{equation}\label{eq:grey_level}
\tau_P P + \tau_N N + \tau_S S,
\end{equation}
qui est une sorte d'échelle de niveaux de gris numérique. 
%Comme pour les scanners abdominaux, l'échelle de Hounsfield est limitée de 
Nous considérons ici une échelle de Hounsfield allant de 
-200 à +200 (en associant ainsi à la valeur -200 la couleur noir et à +200 le blanc) et nous fixons arbitrairement les coefficients ci-dessus à 
\begin{equation}\label{eq:tau_arbitraire}
\tau_P=60,\quad \tau_S=120,\quad \tau_N=-140.
\end{equation}
Cette synthèse d'image scanners sera détaillée un peu plus loin dans ce manuscrit dans le Chapitre~\ref{chap:optim_grey}.


\subsection{Etude approfondie sur un patient spécifique~: \Nber}
\subsubsection{Comparaison des résultats numériques avec les données cliniques}
Nous nous focalisons à présent sur \Nber pour lequel nous avons aussi bien le protocole thérapeutique complet que les données cliniques de l'évolution de l'aire tumorale et la séquence des scanners (présentés dans leur totalité dans l'Annexe~\ref{chap:anx_img_complement}). 
Les simulations numériques sont réalisées dans un domaine carré de taille 
$L=D=6\;cm$ avec 120 points dans chaque direction. 
Le pas de temps~$\Delta t$ est calculé en utilisant l'équation~\eqref{eq:dt}. 


Les paramètres numériques ont été choisis de sorte à reproduire l'évolution de l'aire tumorale. En particulier, nous n'essayons pas de fiter avec les images, nous vérifions seulement que l'évolution spatiale soit plausible comparée aux images médicales. La manière dont ont été obtenus les paramètres est abordée dans la section~\ref{sec:code_modeleEDP} de l'annexe~\ref{chap:structure_code}.  L'ensemble des paramètres est résumé dans la Table~\ref{table_param}.

\begin{table}
\vspace{-1mm}
\rowcolors{4}{white}{\grisclair}
{\footnotesize\smaller[0.5]
\setlength{\doublerulesep}{3pt}
\begin{tabularx}{\textwidth}{|l|X|l|p{18mm}|p{18mm}|}
\hline
%\rowcolor[gray]{0.7}  & & &  Valeur pour & Valeur pour \\
%\rowcolor[gray]{0.7}  & & & le fit de  & le fit de \\
\rowcolor[gray]{0.7} &&& \multicolumn{2}{c|}{\cellcolor[gray]{0.7} Valeur pour le fit de} \\
\cline{4-5}
\rowcolor[gray]{0.7}  Nom & Signification & Unité & \Nber & \Chen \\
\rowcolor[gray]{0.7}  &   &  & (Figure~\ref{fig:fit_area_henbert}) & (Figure~\ref{fig:fit_area_chen})\\
\hline
\input{../data/params_value/table_fit.txt}
$\epsilon_{\xi}$ & Production résiduelle de facteur de croissance & - & 0.1 & 0.1 \\
\hline \hline \hline
$L,D$ & Dimensions du domaine de calcul &$cm$& 6 & 12 \\
$N_x,N_y$ & Nombre de points dans chaque direction du \mbox{domaine} de calcul &-& 120 &132\\
$r_1$ & Rayon de la donnée initiale selon l'axe~$x$  & $cm$ & 0.47 & 0.5 \\
$r_2$ & Rayon de la donnée initiale selon l'axe~$y$ & $cm$ & 0.36 & 0.5 \\
$e$ & Sorte d'excentricité de la donnée initiale &-&0.35&0\\
$\beta$ & Poids du \Twinweno &-&0&0.3\\
\hline
\end{tabularx}
}
\vspace{-1mm}
\caption{\label{table_param}Liste des paramètres du modèle et leur valeurs pour les deux patients considérés -- $d=\textrm{Jour (day)}$
}
\end{table}

L'aire tumorale numérique est ensuite comparée aux aires mesurées sur la Figure~\ref{fig:fit_area_henbert}, page~\pageref{fig:fit_area_henbert}.  
%(les cercles représentent les données réelles, et la ligne pleine représente la simulation numérique). 
Il est important de noter que d'après cette 
figure, l'évolution de l'aire tumorale est correctement reproduite. 


Nous soulignons que la donnée initiale est cruciale dans la croissance de la tumeur. Afin de reproduire qualitativement la forme de la lésion au temps initial (\cf Figure~\ref{fig_hen_spatial_1}), 
la condition initiale est choisie comme une ellipse perturbée. 
Il s'agit là de s'approcher de la forme réelle constatée sur les scanners tout en gardant une géométrie régulière. 
Plus précisément, étant donnés 3 paramètres~$r_1$, $r_2$ et~$e$, nous définissons dans le domaine~$\Omega$, $d(\vecx)$ par
\begin{equation}\label{cond-ini}
d(\vecx)=\sqrt{\left(\dfrac{\mid x - L/2 \mid }{e\mid x - L/2 \mid +
      \crm r_1}\right)^2+\left(\dfrac{\mid y - D/2 \mid }{e\mid y -
      D/2 \mid + \crm r_2}\right)^2 },
\end{equation}
avec~$\vecx=(x,y)$ et où~$c=\dfrac{2\pi}{2\pi-\arccos(1-2\epsilon_{th})}$. 
La fonction~$Y$ definie par\footnote{Notons que si~$e=0$, alors~$Y$ est  invariant par rotation.}
\begin{equation}Y(\vecx)=\left\{
\begin{array}{ll}
1 & \textrm{si}\ d(\vecx)\leq 0.5,\\
0 & \textrm{si}\ d(\vecx)\geq 1, \\
\dfrac{1}{2}\Big(1-\cos \big(2\pi d(\vecx)\big)\Big)& \textrm{sinon},
\label{eq:CI_forme}
\end{array}\right.
\end{equation}
est ensuite utilisée afin d'imposer la condition initiale~:
\begin{equation}
\label{eq:CI}
\begin{aligned}
P_1|_{t=0}&=(1-\Sigma_{ini})Y, \quad
P_2|_{t=0}=\dfrac{\Sigma_{ini}}{1+q_{ini}}Y, \quad
P_3|_{t=0}=q_{ini}P_2(t=0), \\
N|_{t=0}&=0, \qquad \qquad \quad
M|_{t=0}=2\Ms,
\end{aligned}
\end{equation}
où~$\Sigma_{ini}$ représente la proportion de cellules résistantes au traitement~$\TI$ et~$q_{ini}$ est la proportion de cellules~$P_2$
qui sont aussi résistantes au traitement~$\TS$. Pour \Nber, nous avons choisi~$e=0.35$ et~$r_1=0.47$; $r_2=0.36$.

%%%%%%%%%%% Simu scan NBER
\begin{figure}[h]
\subfloat[Jour 119]{\label{fig_nber_simu_spatial_1}
\includegraphics[width=0.32\textwidth]{fit_henbert_form3/vue_scan/vue_scan10.png}}
\subfloat[ Jour 409]{\label{fig_nber_simu_spatial_2}
\includegraphics[width=0.32\textwidth]{fit_henbert_form3/vue_scan/vue_scan32.png}}
\subfloat[Jour 778]{\label{fig_nber_simu_spatial_3}
\includegraphics[width=0.32\textwidth]{fit_henbert_form3/vue_scan/vue_scan60.png}}\\
\subfloat[Jour 870]{\label{fig_nber_simu_spatial_4}
\includegraphics[width=0.32\textwidth]{fit_henbert_form3/vue_scan/vue_scan67.png}}
\subfloat[Jour 961]{\label{fig_nber_simu_spatial_5}
\includegraphics[width=0.32\textwidth]{fit_henbert_form3/vue_scan/vue_scan74.png}}
\subfloat[Jour 1120]{\label{fig_nber_simu_spatial_6}
\includegraphics[width=0.32\textwidth]{fit_henbert_form3/vue_scan/vue_scan86.png}}
\caption{Simulation numérique pour \Nber~: évolution spatiale de la lésion avec reconstruction numérique de scanners.} \label{fig:simu_henbert_scan}
\end{figure}


L'aspect spatial des simulations numériques présentées dans la Figure~\ref{fig:simu_henbert_scan}, fait apparaître les faits suivants~:
\begin{myenumerate}[label=\roman*)]
\item \label{ini} Pendant la phase sans traitement, du jour~0
    au jour~119, la tumeur croît. Comme des cellules nécrosées et proliférantes sont présentes, les images numériques sont hétérogènes, comme on peut le constater Figure~\ref{fig_nber_simu_spatial_1}.
\item \label{TI} Ensuite le traitement~$\TI$ est administré du jour~119 au jour~867.
 Les cellules proliférantes~$P_1$ sont tuées et la nécrose devient prédominante. A cause du choix du coefficient~$\tau_N$, la tumeur numérique devient homogène et plus sombre comme le montre la Figure~\ref{fig_nber_simu_spatial_2}.
\item \label{failureTI} Le regain d'activité cellulaire au jour~776, juste avant la rechute, est caractérisé par une augmentation de l'hétérogénéité~: une couronne proliférante apparaît et recolonise progressivement l'intérieur nécrosé de la tumeur comme l'illustre la  Figure~\ref{fig_nber_simu_spatial_3}. Il est important de noter que le traitement~$\TI$ est encore administré et donc que les cellules résistantes commencent à devenir prédominantes.
\item \label{TS} Ensuite le traitement~$\TS$ est administré du jour~867 au jour~1298. Une fois de plus, la population nécrosée augmente et la tumeur numérique s'assombrit, comme le montrent les Figures~\ref{fig_nber_simu_spatial_4}-\ref{fig_nber_simu_spatial_5}.
\item \label{last} Finalement, au jour~1116, une nouvelle rechute se prépare. Ici encore, elle est caractérisée par une couronne proliférante au bord de la tumeur (\cf  Figure~\ref{fig_nber_simu_spatial_6}).
\end{myenumerate}

Les résultats numériques spatiaux sont en accord avec les données, au moins jusqu'au début de la dernière rechute autour du jour~1000. En fait, les cinq étapes que nous observons numériquement sont également constatées sur les scanners (\cf Figure~\ref{fig_hen_spatial_1}-\ref{fig_hen_spatial_6}).


Après le jour~1116, notre modèle n'est probablement plus valide car la structure de la tumeur numérique est très différente de celles des scanners. Peut-être que d'autres phénomènes qui ne sont pas pris en compte par notre modèle se produisent, comme des changements microenvironementaux ou d'autres mutations cellulaires. Des interactions dans la~$3^\textrm{ième}$ direction (non prise en compte ici, le modèle étant 2D) peuvent aussi être impliquées.

%\begin{figure}[h]
%\includegraphics[width=0.32\textwidth]{fit_henbert_form3/P1.png}
%\includegraphics[width=0.32\textwidth]{fit_henbert_form3/P2.png}
%\includegraphics[width=0.32\textwidth]{fit_henbert_form3/P3.png}\\
%\includegraphics[width=0.32\textwidth]{fit_henbert_form3/N.png}
%\includegraphics[width=0.32\textwidth]{fit_henbert_form3/xi.png}
%\includegraphics[width=0.32\textwidth]{fit_henbert_form3/total.png}
%\caption{Evolution de la masse (intégrale du niveau de gris, dans une unité arbitraire) et de l'aire ($mm^2$) de chaque population de cellules et évolution du signal angiogénique ($cm.d^{-1}$) données par les simulations numériques.} \label{fig:simu_henbert_mass_area}
%\end{figure}

%\begin{figure}[h!]
%%\begin{figure}[h]
\begin{figure}[p]
\includegraphics[width=0.45\textwidth]{fit_henbert_form3/P1.png} \qquad
\includegraphics[width=0.45\textwidth]{fit_henbert_form3/P2.png} \\ \\
\includegraphics[width=0.45\textwidth]{fit_henbert_form3/P3.png} \qquad
\includegraphics[width=0.45\textwidth]{fit_henbert_form3/N.png} \\ \\
\includegraphics[width=0.45\textwidth]{fit_henbert_form3/xi.png} \qquad
\includegraphics[width=0.45\textwidth]{fit_henbert_form3/total.png}
\caption{Evolution de la masse (intégrale du niveau de gris, dans une unité arbitraire) et de l'aire ($mm^2$) de chaque population de cellules et évolution du signal angiogénique ($cm.d^{-1}$) données par les simulations numériques.} \label{fig:simu_henbert_mass_area}
\end{figure}

Nous soulignons que le modèle semble fournir d'importantes informations auxquelles les cliniciens n'ont pas accès avec les dispositifs d'imagerie. Plus précisément, la  Figure~\ref{fig:simu_henbert_mass_area} qui présente l'évolution de l'aire et de la masse  de  \emph{chaque} population de cellules, nous permet de dire que selon notre modèle~: 
\begin{myitemize}
\item Pendant le premier rétrécissement de la tumeur causé par le 
traitement~$\TI$, du jour~119 au jour~406, on observe que
\begin{myenumerate}[label=\roman*)]
\item Le traitement~$\TI$ tue les cellules proliférantes~$P_1$ qui deviennent du tissu nécrosé. 
\item La réduction de l'aire est due à l'élimination de la nécrose par le système immunitaire. 
\item Les populations de cellules~$P_2$ et~$P_3$ qui ne sont pas sensibles au traitement~$\TI$ continuent de se diviser.
\end{myenumerate}
\item Pendant plusieurs mois, du jour~406 au jour~778, l'aire tumorale continue de décroître lentement à cause de la mort de la population~$P_1$. Cependant les populations de cellules~$P_2$ et~$P_3$ continuent de grandir et prennent progressivement la place des cellules nécrosées éliminées.  Même si l'activité cellulaire de~$P_2$ et~$P_3$ n'est pas affectée par le traitement, il conduit dans un premier temps à une stabilisation de l'aire tumorale avant la recroissance au 
jour~778. En fait, quand~$P_2+P_3$ devient trop grand, la croissance de l'aire tumorale reprend et est gouvernée par les cellules résistantes au traitement~$\TI$.
\item Pendant le traitement~$\TS$, du jour~867 au jour~1298, nous pouvons remarquer que 
\begin{myenumerate}[label=\roman*)]
\item La population~$P_1$ est encore sensible au traitement. 
\item La population~$P_2$, sensible au traitement~$\TS$, se nécrose.
\item La population~$P_3$ qui est résistante aux deux traitements continue de croître.
\end{myenumerate}
\end{myitemize}

Il est important de noter que pour chaque rechute, l'activité proliférante reprend sur le pourtour de la tumeur, là où sont présents les nutriments. 
De plus, notre modèle présente des différences entre l'évolution de l'aire tumorale~$\mathcal{A}$ et l'évolution de la masse tumorale~$\mathcal{M}$ données respectivement par~\eqref{eq:airetot} et~\eqref{eq:masstot}. 

En particulier, la masse de cellules cancéreuses décroît juste après l'administration du traitement alors que l'aire tumorale décroît avec un retard. 
Ce délai est dû au fait que les cellules proliférantes tuées basculent dans une phase nécrosée. Ainsi l'aire tumorale reste la même jusqu'à ce que le système immunitaire élimine cette nécrose. 
De plus, notre modèle permet de distinguer les effets des deux traitements sur l'angiogenèse. 
En effet, le traitement~$\TS$ inhibe directement le signal angiogénique alors que le  traitement~$\TI$ tue la population~$P_1$, ce qui implique indirectement une diminution de la production du signal. 
Notre modèle permet donc de décrire un mécanisme plausible de réponse tumorale face à un traitement. 

\subsubsection{Etude numérique de l'influence de l'efficacité du traitement~$\TI$}\label{subsec:NumEff}
Nous nous intéressons ici, à l'étude numérique de différents aboutissements obtenus avec le 
traitement~$\TI$ afin d'examiner son influence sur la croissance tumorale. 
Prenons les paramètres de la Table~\ref{table_param} et faisons juste varier~$\muI$. 
Définissons deux durées caractéristiques~:
\begin{myitemize}
\item $\TPFS$ (progression-free survival time) qui est le temps de survie sans aggravation. C'est le temps pour lequel la taille de la tumeur reste inférieure à sa taille au début du traitement.  
\item $\Td$, qui est le temps de doublement. C'est la durée pour laquelle l'aire tumorale reste inférieure au double de sa taille initiale au début du traitement. 
\end{myitemize}
 %%%%%%% Figure variation dose traitement
\begin{figure}[!htb]
\centering
\captionsetup[subfigure]{margin=10pt}
\subfloat[Temps de survie sans progression ($\TPFS$ en jours) en fonction de~$\muI$.]{
\label{fig:Tmaintien_nber}
\includegraphics[width=0.5\textwidth]{DB-nber-eff-glivec/Tmaintien.png}}
\subfloat[Temps de doublement ($T_{double}$ en jours) en fonction de~$\muI$.]{
\label{fig:Tdouble_nber}
\includegraphics[width=0.5\textwidth]{DB-nber-eff-glivec/Tdouble.png}}\\
\subfloat[Aire minimale atteinte ($\Amin$ en~$mm^2$) en fonction de~$\muI$.]{
\label{fig:Amin_nber}
\includegraphics[width=0.5\textwidth]{DB-nber-eff-glivec/Amin.png}} 
\subfloat[Portrait de phase.]{
\label{fig:portrait_phase_nber}
\includegraphics[width=0.5\textwidth]{DB-nber-eff-glivec/portrait_phase.png}}
\caption{Efficacité du traitement~$\muI$ sur \Nber. Les étoiles correspondent aux paramètres utilisés dans la Figure~\ref{fig_nber} pour le fit de l'aire tumorale. \label{fig:eff_glivec_nber}}
\end{figure}


Notons que cette étude est purement théorique et ne peut pas être utilisée telle quelle pour l'optimisation de traitement car l'évolution complète de la maladie (phases de rechute incluses) est nécessaire pour calibrer les paramètres utilisés dans les simulations. Par conséquent, il est clair que notre approche ne peut pour l'instant pas conduire à une détermination d'un protocole optimal de traitement. 
%, mais cette étude numérique est déjà un premier pas crucial au vue de l'importance de l'enjeu.


Sur la Figure~\ref{fig:Tmaintien_nber} est présenté le temps de survie sans aggravation~$\TPFS$ en fonction de~$\muI$. 
Si~$\muI$ est en dessous une valeur seuil~$\mu_{th}$ ($\mu_{th}\sim 0.0047$ pour \Nber), alors la croissance tumorale n'est pas stoppée.  
Pour~$\muI$ en dessus de ce seuil, $\TPFS$ augmente rapidement et atteint un plateau, ce qui signifie qu'il n'est pas nécessaire d'augmenter la dose puisque cela n'aurait pas d'effet sur  le temps de survie sans aggravation~$\TPFS$.

Sur la Figure~\ref{fig:Tdouble_nber} est présenté le temps de doublement~$\Td$. 
Comme nous pouvons le voir, $\Td$ n'augmente pas avec l'augmentation de la dose au-delà du seuil~$\mu_{th}$, ce qui signifie une fois de plus qu'il n'est pas nécessaire d'augmenter la dose au-delà d'une certaine valeur seuil.


L'aire minimale atteinte par la tumeur en fonction de~$\muI$ est présentée sur la Figure~\ref{fig:Amin_nber}. Notons que la courbe est décroissante~: en effet plus~$\muI$ est élevé, plus l'aire minimale atteinte est petite. Cependant, comme le montre la  Figure~\ref{fig:portrait_phase_nber}, la relation entre la taille minimale et le temps de doublement n'est pas monotone. En particulier, si la taille minimale est très petite, alors le temps de doublement peut être plus petit.  
Ceci peut être dû à des mécanismes de sélection darwinienne~: 
la population~$P_1$, qui est prédominante lorsque la lésion est détectée, est tuée plus vite par le traitement et ainsi plus de place et de nutriments sont disponibles pour la croissance des populations résistantes~$P_2$ et~$P_3$. Ainsi le temps de doublement devient plus court car l'émergence de ces populations résistantes est plus rapide. 
Ces courbes montrent qu'il existe un seuil~$\mu_{th}$ à partir duquel le traitement~$\treat_1$ est efficace. Au-delà de ce seuil, l'aire minimale atteinte par la tumeur décroît, mais le temps de survie sans progression du patient n'augmente pas.

\section{Discussion}
Dans la section précédente, nous avons étudié en détail notre modèle sur un patient spécifique, \Nber, dont la lésion tumorale a été suivie par une série de scanners. Nous avons trouvé les paramètres qui rendent possible la comparaison qualitative des images médicales avec les images numériques, grâce à notre synthèse numérique de scanners et nous avons également fité l'aire tumorale. La lésion du patient est intéressante car elle est confinée à l'intérieur du foie et donc son évolution n'est pas mécaniquement contrainte par l'organe. 
Cependant, dans certains cas, la métastase est proche du bord du foie. Dans de tels cas, il n'y a pas d'espoir de produire des résultats quantitatifs sur l'évolution spatiale de la lésion, mais dans la section suivante, nous montrerons que l'évolution de l'aire tumorale peut être correctement décrite.
\subsection{Evolution de l'aire tumorale de \Chen}\label{section chen}
Dans cette section, nous nous intéressons à \Chen, dont l'évolution tumorale est assez différente de celle \Nber. En effet le protocole clinique associé à ce patient est le suivant~:
\begin{myenumerate}[label=\roman*)]
\item \Chen est traité dans un premier temps avec un inhibiteur de tyrosine kinase mono-cible (imatinib), qui stabilise l'aire de la tumeur pendant plus de 10 mois avant une rechute. 
\item Le sunitinib, inhibiteur de tyrosine kinase multi-cibles, est démarré mais contrairement à \Nber, il est totalement inefficace. 
\item Un troisième traitement est administré. Le sorafenib est un autre inhibiteur de tyrosine kinase multi-cibles, qui a à la fois un effet cytotoxique et antiangiogénique. La tumeur semble être sensible à ce médicament pendant plusieurs mois jusqu'à ce qu'une nouvelle rechute arrive. 
\end{myenumerate}
Malheureusement, comme illustré par la Figure~\ref{fig_chen_spatial}, l'évolution tumorale est très rapide entre le scanner du jour~429 et celui de jour~845 sur lequel la métastase a atteint le bord du foie et ainsi, avec notre modèle nous n'avons pas espoir de fournir une croissance tumorale numérique qui soit spatialement en accord avec les scanners car les effets de bord de l'organe devraient être pris en compte. Par conséquent, nous nous intéresserons à  l'aire tumorale comme donnée par les cliniciens, le principal enjeu ici étant de capter une croissance aussi rapide. 


Comme le sunitinib est totalement inefficace dans ce cas, nous considérons que le 
traitement~$\TI$ est administré du jour~0 au jour~845, tandis que le 
traitement~$\TS$, basé sur le sorafinib, est administré du jour~1049 au jour~1600. 
Nous avons trouvé des paramètres qui rendent possible la comparaison qualitative entre l'aire tumorale mesurée par les cliniciens et notre aire tumorale numérique. Ces paramètres sont donnés dans la  Table~\ref{table_param}. 
La donnée initiale a été choisie invariante par rotation, puisque nous ne pouvons pas %en accord avec les données de la Figure~\ref{fig_chen_spatial_1}.
ici reproduire la forme de la métastase qui atteint rapidement le bord du foie.
Comme on peut le voir, l'évolution tumorale est très raide~: entre le  jour~416 et le jour~614 l'aire de la tumeur a été multipliée par plus de 9 et une fois que le sorafenib est administré, l'aire décroît de~4\;500\;$mm^2$ à~2\;850\;$mm^2$. Nous parvenons malgré tout à obtenir, au moins qualitativement, un tel comportement avec notre modèle, même si le fit n'est pas parfait. 

\begin{figure}[t]
\captionsetup[subfigure]{margin={7pt,7pt}}
\begin{tabular}{@{\hspace{-1mm}}ccc}
\setlength{\unitlength}{0.01\textwidth}
\subfloat[23 mai 2007 -- Jour 0]{\label{fig_chen_spatial_1}
\begin{picture}(20,20)
\put(0,0){\includegraphics[width=0.2\textwidth]{scan_chen/2007_mai_23.jpg}}%
\linethickness{.8mm}
\put(5,5){\vector(1,3){2.5}}
\end{picture}
}&
\subfloat[25 juill. 2008 -- Jour 429]{\label{fig_chen_spatial_2}
\setlength{\unitlength}{0.01\textwidth}
\begin{picture}(20,20)
\put(0,0){\includegraphics[width=0.2\textwidth]{scan_chen/2008_juil_25.jpg}}%
\linethickness{.8mm}
\put(3.9,4.2){\vector(1,3){2.5}}
\end{picture}
}&\\
\subfloat[14 sept. 2009 -- Jour 845]{\label{fig_chen_spatial_3}
\includegraphics[width=0.2\textwidth]{scan_chen/2009_sept_14.jpg}%
}&
 \subfloat[6 avril 2010 -- Jour 1049]{\label{fig_chen_spatial_4}
\includegraphics[width=0.2\textwidth]{scan_chen/2010_avril_06.jpg}%
 }&\\
 \subfloat[28 sept. 2010 -- Jour 1224]{\label{fig_chen_spatial_5}
\includegraphics[width=0.2\textwidth]{scan_chen/2010_sept_28.jpg}%
 }&
 \subfloat[20 mai 2011 -- Jour 1458]{\label{fig_chen_spatial_6}
\includegraphics[width=0.2\textwidth]{scan_chen/2011_mai_20.jpg}%
 }&
\multirow{-3}{*}[105mm]{
\subfloat[Aire tumorale. Chaque point représente l'aire de la tumeur mesurée sur les scanners et la ligne représente notre résultat numérique. \newline
Les lettres font référence aux scanners montrés sur la gauche. Les symboles 
 $\oplus$ représentent les tumeurs hétérogènes, {\large$\bullet$} les tumeurs plutôt homogènes et {\large$\circ$} les tumeurs pour lesquelles il est difficile de détecter sur le scanner un aspect hétérogène ou homogène dominant.  
Les valeurs des paramètres utilisés dans la simulation numérique sont données dans la Table~\ref{table_param}.]{\label{fig:fit_area_chen} 
 \includegraphics[width=0.5\textwidth]{chen_area.png}
 }}
\end{tabular}
\caption{\label{fig_chen_spatial} Evolution spatiale de la métastase hépatique de \Chen sur une série de scanners. Nous examinons l'évolution de la lésion\protect\footnotemark~située au départ en haut à gauche de la Figure~\ref{fig_chen_spatial_1}. Ici n'est présenté qu'une sélection de scanners~: l'ensemble complet est présenté dans l'annexe~\ref{chap:anx_img_complement}.}
\end{figure}
\footnotetext{Ne pas confondre la métastase avec la vésicule biliaire qui est plus grosse sur les deux premiers scanners.}


Nous avons aussi examiné l'effet de l'efficacité du traitement~$\treat_1$, comme pour \Nber. 
Nous avons pu observer que le temps de survie sans aggravation~$\TPFS$, le temps de doublement~$\Td$ et l'aire minimale atteinte par la  lésion~$\Amin$ ont le même profil que pour le patient étudié dans la section précédente. En particulier, il existe un seuil~$\mu_{th}$ en dessous duquel le  traitement n'a aucun effet sur~$\TPFS$. Ensuite une augmentation de la dose n'améliore pas non plus~$\TPFS$ qui atteint un plateau au-delà de~$\mu_{th}$ comme pour \Nber. 
Le temps de doublement~$\Td$ en fonction de~$\muI$ est également monotone, contrairement à~$\Amin$. Ainsi, comme précédemment, au-delà du seuil~$\mu_{th}$, le minimum de l'aire tumorale décroît encore, mais le temps de survie globale du patient n'est pas augmenté.



\subsection{Consistance du modèle\label{section_possibilites}}
\begin{figure}[h]
\vspace{-1cm}
\centering \hfill
\captionsetup[subfigure]{margin=10pt}
\subfloat[Imatinib à partir du jour 119]{
\label{fig:possibilites1}
\includegraphics[width=0.3\textwidth]{possibilites/glivec_efficace.png}
}
\renewcommand{\thesubfigure}{b-c-d}
\subfloat[Imatinib à partir du jour 119 jusqu'au jour 867 puis sunitinib juste après ]{
\label{fig:possibilites2}
\includegraphics[width=0.3\textwidth]{possibilites/glivec_rechute.png}
}
\renewcommand{\thesubfigure}{e-f-g}
\subfloat[Imatinib à partir du jour 119 jusqu'au jour 300 puis sunitinib juste après ]{
\label{fig:possibilites3}
\includegraphics[width=0.3\textwidth]{possibilites/glivec_inefficace.png}
}
\caption{\label{fig:possibilites}Différents comportements pris en compte par notre modèle.}
\vspace{-1cm}
\end{figure}

Notre modèle reproduit les données cliniques pour \Nber et \Chen. De plus, il est capable de traiter l'ensemble des comportements rapportés par les médecins comme le montre la Figure~\ref{fig:possibilites} 
(\cf Table~\ref{table_param_possibilites} dans l'Annexe~\ref{chap:anx_img_complement} page~\pageref{table_param_possibilites}, pour les différentes valeurs des paramètres utilisés):
\begin{myenumerate}[label={\alph{*})}]
\item La métastase est contrôlée par le traitement~$\TI$ (imatinib). 
Dans ce cas, il n'y a pas besoin de changer de traitement. 
\item La métastase est contrôlée par le traitement~$\TI$ mais la tumeur recroît ensuite. Le traitement~$\TS$ (sunitinib ou sorafenib) est alors administré avec succès et l'aire tumorale est contrôlée. 
\item La métastase est contrôlée par le traitement~$\TI$ avant une première rechute. Le traitement~$\TS$ (sunitinib ou sorafenib) est alors administré et est efficace avant une seconde rechute. 
\item La métastase est contrôlée par le traitement~$\TI$ avant une première rechute. Ensuite le traitement~$\TS$ est totalement inefficace. 
\item Le traitement~$\TI$ est totalement inefficace. 
Ensuite le traitement~$\TS$ est efficace et l'aire de la tumeur est contrôlée. 
\item Le traitement~$\TI$ est totalement inefficace. 
Ensuite le traitement~$\TS$ est efficace avant une rechute.
\item Le traitement~$\TI$ est totalement inefficace. 
Ensuite le  traitement~$\TS$ est totalement inefficace. 
Ce type de profil est représentatif des patients avec une mutation génétique EXON, comme rapporté par Andersson
{\it et al.}~\cite{Andersson2006}, ou Hirota {\it et al.}~\cite{Hirota1998} et comme présenté dans le chapitre précédent sur la Figure~\ref{fig:patients_EXON}, page~\pageref{fig:patients_EXON}. 
\end{myenumerate}

\FloatBarrier
\section{Conclusion}

Nous avons fourni ici un modèle dépendant de chaque patient, basé sur des EDPs, qui décrit le comportement global de métastases hépatiques de GISTs durant les différentes étapes. 


Le modèle a été numériquement comparé avec les observations cliniques concernant \Nber, qui a été traité successivement à l'imatinib puis au sunitinib.
Comme présenté sur la 
Figure~\ref{fig:fit_area_henbert}, notre modèle fournit des résultats qui sont qualitativement en accord avec les données cliniques. En particulier, notre modèle est capable de décrire non seulement l'évolution de la taille de la lésion, mais aussi sa structure, comme l'illustrent les Figures~\ref{fig_nber} et~\ref{fig:compare_spatial_nber}.

Il est intéressant de noter que nos simulations numériques font apparaître une couronne de cellules proliférantes sur le pourtour de la tumeur juste avant la rechute. Ceci semble corroboré dans les scanners par l'augmentation de l'\hetero tumorale, au sens des niveaux de gris, avant la rechute. Plus la métastase est hétérogène, plus la rechute est imminente. Ce résultat souligne le fait que le critère RECIST n'est pas suffisant dans ce cas pour évaluer l'efficacité d'un traitement.


Nous avons également étudié numériquement l'effet du paramètre~$\muI$, lié à l'efficacité du traitement~$\TI$. Nous avons montré que, selon notre modèle, augmenter la valeur de~$\muI$ -- ce qui peut être vu comme une augmentation de la dose administrée -- ne fournit pas de meilleurs résultats en terme de survie sans progression comme le montre la Figure~\ref{fig:eff_glivec_nber}.


Il est important de noter que notre modèle fite bien les données, cependant il n'est pas prédictif. En effet, comme expliqué par la Figure~\ref{fig:possibilites2}, on peut voir que les trois comportements typiques~$\rm{b,c,d}$ peuvent être obtenus avec le même comportement durant les 400 premiers jours. Ceci est dû au fait que sur les images médicales, il est compliqué de dire {\it a priori}, ce qu'il va se passer lors du second traitement avant même de l'avoir utilisé. 
%avant d'utiliser le second traitement, entre 
Entre une tumeur principalement composée de la population~$P_2$ (comme pour le cas~${\rm b}$) de la Figure~\ref{fig:possibilites2}) et une tumeur avec une population~$P_3$ (comme donnée par la courbe~${\rm d}$) de la Figure~\ref{fig:possibilites2}), les issues à long terme sont très différentes. 
En d'autres termes, on peut trouver différents jeux de paramètres qui donnent le même comportement initial. Ceci signifie que des données plus précises, comme de l'imagerie fonctionnelle, devront être nécessaires pour une meilleure analyse de la structure interne de la métastase. 


%En conclusion, nous avons fourni un modèle qui fite avec les scanners cliniques que nous avons. De futurs travaux consisteront à ajouter plus d'informations biologiques que nous ne pouvons pas obtenir depuis les scanners, afin d'obtenir un modèle prédictif. L'imagerie fonctionnelle (TEP-scan ou IRM) ou des biopsies pourraient enrichir le présent modèle.


Dans le chapitre suivant, nous allons présenter un peu plus en détail un nouveau type de schéma WENO5, le \twinweno, utilisé dans nos simulations numériques. Une instabilité se créant avec les schémas classiques, ce schéma la corrigera.


\end{document}