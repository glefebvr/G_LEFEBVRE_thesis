\documentclass[main.tex]{subfiles}

\begin{document}
\specialchap{Glossaire}

\begin{description}
\item[Angiogenèse] Processus de création de nouveaux vaisseaux sanguins. 
\item[Antiangiogénique] (ou anti-angiogénique) Médicament qui inhibe l'angiogenèse.
\item[Apoptose] Processus d'auto-destruction cellulaire.
\item[Chimiotactisme] Phénomène par lequel des cellules dirigent leurs mouvements selon le gradient de concentration d'une certaine molécule.
\item[Cytotoxique] Qui détruit les cellules.
\item[Développement thérapeutique] Série de traitements utilisés. La nécessité de changer de traitement intervient lors des rechutes successives. 
\item[Endothélial] Relatif aux cellules qui recouvrent la paroi intérieure des vaisseaux sanguins. Leur multiplication et leur maturation permettent l'angiogenèse.
\item[Epithélium] Tissus constitués de cellules étroitement juxtaposées (ou jointives), sans interposition de fibre ou de substance fondamentale. Ces tissus servent notamment de paroi entre deux milieux. 
\item[Extravasion] Processus par lequel les cellules invasives de tumeur primaire peuvent traverser les membranes pour créer des métastases. 
\item[Facteur de croissance] Protéine externe à la cellule commandant sa croissance ou sa duplication.
\item[Hypoxie] Manque d'oxygène.
\item[Imatinib] Thérapie ciblée à effet cytotoxique. 
\item[Indifférenciation] (cellulaire) Retour à un état proche de la cellule souche. 
\item[Quiescence] Etat de dormance dans lequel entre une cellule lorsque les conditions environnementales tendent à devenir défavorables.
\item[Métastase] Tumeur distante provenant d'une tumeur primaire.
\item[Néovascularisation] Réseau vasculaire créé par la tumeur afin de l'alimenter.
\item[Réhaussement] (d'un scanner) Eclaircissement des niveaux de gris présent sur les scanners du à l'injection d'un produit de contraste. 
\item[Stroma] Tissu de soutien, non impliqué dans la fonction première de l'organe.
\item[Sunitinib] Thérapies ciblée à effet cytotoxique et antiangiogénique. 
\item[Temps artériel] Temps après lequel un produit injecté en intraveineuse met pour parvenir au foie par l'artère hépatique.
\item[Temps portal] Temps après lequel un produit injecté en intraveineuse met pour parvenir au foie par la veine porte.
\item[Thérapie ciblée] Médicament ciblant un type spécifique de voie moléculaire, généralement caractéristique des cellules malignes. 
\item[Tomodensitométrie] Taux d'absorption aux rayons X. Synonyme de scanographie. 
\item[Tyrosine kinase] Enzyme commandant de nombreuses fonctions cellulaires (dont pour certaines, le processus de division).
\end{description}
\end{document}