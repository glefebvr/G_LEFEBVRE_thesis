\documentclass[main.tex]{subfiles}
%\pagestyle{main}

\begin{document}
\specialchap{Introduction}
%\chapter{Introduction}

%\lettrine{L}{e} cancer est l'un des plus grand fléau du XXI\ieme %\up{ième} 
%siècle.  Dans les pays industrialisés, il est la deuxième cause de mortalité derrière les maladies cardio-vasculaires. 
\lettrine{L}{e} cancer est, dans les pays industrialisés, la deuxième cause de mortalité derrière les maladies cardio-vasculaires. 
%% http://www.lexpress.fr/actualite/societe/sante/le-cancer-un-fleau-en-plein-essor_1552976.html
En 2012, l'OMS\footnote{Organisation Mondiale de la Santé} attribuait à cette maladie approximativement 14 millions de nouveaux cas et 8,2 millions de décès~\cite{bernard2014world}. 


\paragraph{}
De nombreuses études fondamentales, pré-cliniques ou cliniques ont été réalisées afin de mieux comprendre les mécanismes impliqués dans la genèse et le développement du cancer. De même, de nombreux travaux ont permis d'apprécier la réponse aux traitements. Toutefois, il persiste encore beaucoup d'interrogations sur cette maladie. De nouvelles méthodes d'exploration et de compréhension sont donc nécessairement à développer.
% Bien que de nombreuses recherches soient financées sur le sujet, beaucoup d'interrogations règnent encore sur cette maladie. 
 D'un autre côté, les mathématiques ont pris une place omniprésente dans la vie de tous les jours au travers de la modélisation~: aéronautique, aérospatial, automobile, économie, télécommunication... Elles sont capables de modéliser un très large spectre de problèmes physiques, chimiques, biologiques, épidémiologiques, etc. Pourquoi ne pas les utiliser pour modéliser la croissance tumorale, en particulier pour prédire les rechutes aux traitements ou pour personnaliser les traitements à chaque patient ? 
 
\paragraph{}
Les travaux présentés ici s'inscrivent dans cette démarche~\cite{cornelis2013vivo,jouganous2014}. Ils constituent un premier pas dans la modélisation de résistance aux traitements, basée sur des images cliniques. 
Plus particulièrement, il s'agit là de proposer un modèle mathématique décrivant la croissance de métastases 
%(tumeurs filles) 
(localisations secondaires disséminées à partir d'une lésion primitive)
hépatiques en provenance d'une  
tumeur stromale\footnote{Le stroma est un tissu de soutien, non impliqué dans la fonction première de l'organe.} de l'intestin grêle
%cancer du stroma\footnote{Tissu de soutien, non impliqué dans la fonction première de l'organe.} gastro-instestinal  
(GIST).
Ce type de cancer touche de 9 à 14 personnes sur un million par an~\cite{Nilsson2005}. 
il s'agit donc d'un cancer assez rare mais il présente des spécificités qui en font un modèle intéressant pour l'étude de résistances aux traitements. 
Dans 25\% des cas, ce cancer migre vers le foie~\cite{dematteo2000}. 
%%Ces métastases sont connues pour être relativement résistantes aux traitements anti-cancéreux, ce qui donne un intérêt certain à la modélisation de leurs comportements. 
%Ces métastases peuvent être résistantes aux traitements conventionnels comme la chimiothérapie ou la radiothérapie. Ainsi d'autres thérapeutiques ont été proposées comme les inhibiteurs de la tyrosine kinase ou les traitements anti-angiogéniques.  Malheureusement, en fonction des mutations rencontrées ou acquises, des résistances à ces traitements peuvent également être rencontrées sur ce type de métastases, ce qui donne un intérêt certain à la modélisation de leurs comportements. 
En fonction des mutations rencontrées ou acquises, ces métastases peuvent être résistantes aussi bien aux traitements conventionnels (comme la chimiothérapie ou la radiothérapie) qu'aux nouvelles thérapies ciblées (comme les inhibiteurs de la tyrosine kinase ou les traitements anti-angiogéniques), ce qui donne un intérêt certain à la modélisation de leurs comportements. 
Notre but ici, est de construire un modèle qui reproduise qualitativement et quantitativement la croissance d'une tumeur, et ce à partir de l'imagerie médicale.

\paragraph{}
Le suivi clinique des patients sous traitement est assuré par la réalisation de scanners réguliers, réalisés généralement tous les deux mois. Ceux-ci constituent la seule et unique source d'informations cliniques dont nous disposons ici.  
A l'heure actuelle, pour mesurer l'efficacité d'un traitement, on %le corps médical 
utilise le critère RECIST\footnote{de l'anglais~: Response Evaluation Criteria In Solid Tumors}, qui consiste à ne retenir %, des images médicales, 
que le diamètre de la plus large lésion. 
L'information exploitée %retenue 
semble donc très faible au regard de toutes celles contenues dans les scanners. 
Plusieurs études ont déjà montrées ses limitations~\cite{benjamin2007we}, en particulier lors de traitements anti-angiogéniques. L'imagerie fait apparaître beaucoup d'autres particularités, notamment l'\hetero tumorale~\cite{heppner1984tumor,chabat2003obstructive} à laquelle nous nous intéresserons tout au long de ce manuscrit. L'\hetero peut-elle être liée à l'émergence de populations cellulaires résistantes (comme dans le cas des sarcomes~\cite{eary2008spatial}) ? Peut-on reproduire ce caractère par la modélisation ? Comment la quantifier ?
%La principale particularité à laquelle nous nous intéresserons sera l'\hetero tumorale. Quelles sont les informations sur le comportement de la tumeur que l'ont peut extraire de l'imagerie médicale ? Les critères actuels de lecture de l'imagerie médicale sont-ils suffisant ? L'\hetero peut-elle apportée une information supplémentaire ? 


%%Son but est de fournir un modèle spatial de GIST pour les cas traités cliniquement de manière standard afin de comparer le modèle avec les images, et éventuellement de souligner des particularités dans la croissance tumorale ou dans les rechutes.

\paragraph{}
Les interrogations précédemment posées seront abordées dans ce manuscrit au travers de ses différents chapitres. 

\paragraph{}
Le premier chapitre est dédié à la présentation des aspects biologiques et cliniques du cancer.  
Il présente non pas de manière exhaustive tous les mécanismes du cancer, mais ceux nécessaires à la compréhension des choix réalisés dans notre modélisation. 
La croissance d'une tumeur sera d'abord présentée. 
Puis, la manière dont les métastases se disséminent sera abordée. Les différents types de traitements seront ensuite présentés. La seule et unique source d'informations cliniques que nous ayons étant des images scanners, nous nous devons de comprendre comment ce type d'images est acquis. Nous discuterons ensuite du critère RECIST et de ses limitations avant de terminer par les spécificités des métastases hépatiques de GISTs.


\paragraph{}
Dans le second chapitre, nous présentons le modèle, qui consiste en un système d'équations aux dérivées partielles (EDPs) non linéaires afin de prendre en compte les aspects spatiaux de la croissance tumorale. Actuellement, 
les modèles basés sur des équations différentielles ordinaires (EDOs), comme les modèles~\cite{gerlee2013model} de Mendelsohn, de Gompertz ou de 
Bertalanffy, permettent de suivre la croissance de l'aire tumorale mais ils ne considèrent aucun aspect spatial de cette croissance. Nous en référons à l'analyse de Benzekry {\it et al.} pour plus de détails sur de tels modèles 1D~\cite{benzekry2014}. 
Notre modèle est dans le même esprit que celui de Ribba~{\it et
  al.}~\cite{Ribba2006532} -- nous vous renvoyons également à~\cite{Iollo2012,gallinato2014} -- pour décrire l'évolution de la maladie. 
En ce qui concerne notre modèle, il a été choisi de travailler en 2D d'espace dans le but de coller avec les habitudes des radiologistes (le critère RECIST imposant de sélectionner une coupe). De nombreux modèles spatiaux existent déjà~: des modèles basés sur des automates  cellulaires~\cite{alarcon2003cellular,drasdo2003individual}, à ceux basés sur la théorie du mélange~\cite{ambrosi2002}, en passant par les modèles d'agents~\cite{mansury2002emerging} et les modèles basés sur la théorie de réaction-diffusion~\cite{rockne2009mathematical} ou sur la mécanique des fluides~\cite{Iollo2012}. Plusieurs échelles de modélisation sont couvertes par ce panel de modèles~\cite{cristini2010multiscale}. Afin de reproduire les données cliniques, un modèle macroscopique a été choisi ici. 
%%%%% on n'a pas choisi le modele de diffusion car requiert une connaissance au bord de la tumeur, qui ne peut pas etre extraite precisemment des scanners + bord tumoral = notion controversé

La principale nouveauté du modèle réside en la description des traitements. Deux traitements sont considérés~: le premier traitement a un effet cytotoxique alors que le second a un effet à la fois cytotoxique et antiangiogénique. Trois populations de cellules proliférantes sont utilisées pour décrire la résistance aux traitements~: une population de cellules sensible aux deux traitements, une autre sensible uniquement au second traitement et une dernière résistante aux deux traitements.  
Nous incorporons également un modèle simple d'angiogenèse puisqu'elle joue un rôle crucial dans la croissance métastatique.


%Une fois le modèle construit, nous présentons les schémas numériques utilisés pour résoudre les EDPs. 
%En particulier, nous présentons dans la 
%Section~\ref{sec:NumMet} un nouveau type de schéma WENO5 qui stabilise le calcul numérique en utilisant une combinaison du stencil classique du WENO5 avec un autre stencil tourné par rapport au premier. 

Dans la Section~\ref{sec:NumRes}, nous comparons dans le détail les résultats numériques fournis par notre modèle aux données cliniques pour un patient donné, pour lequel nous disposons de l'ensemble du protocole clinique. 
%Mentionnons qu'il a été nécessaire d'introduire une reconstitution numérique de scanner depuis les résultats numériques afin de comparer les niveaux de gris des scanners avec nos simulations. 
Une fois le fit obtenu, nous examinons numériquement l'effet de la dose de traitement sur la progression de la croissance tumorale. Un résultat contre-intuitif est ainsi obtenu~: avec les paramètres utilisés pour le fit des données, une augmentation de la dose du premier traitement n'améliore pas le temps de survie sans aggravation. Ce résultat est expliqué dans la Section~\ref{subsec:NumEff}. 
Enfin, nous discuterons sur la consistance de notre modèle en présentant les différents comportements d'évolution tumorale que l'on peut obtenir. 
Nous fitons également l'aire tumorale d'un autre patient dont la tumeur est proche du bord du foie. 
Pour un tel patient, la forme de la tumeur ne pourra pas être retrouvée car notre modèle ne prend pas en compte les contraintes mécaniques correspondantes. Cependant, l'aire tumorale semble bien reproduite par nos simulations. 
Les résultats de ce second chapitre sont un premier pas vers la compréhension des résistances des métastases hépatiques de GIST aux médicaments. Cependant, comme mentionné dans la conclusion, les scanners ne fournissent pas suffisamment d'informations pour permettre à notre modèle d'être prédictif~: évaluer les paramètres sur les scanners réalisés durant la croissance avant traitement et durant la première ligne de traitement ne suffit pas pour prédire la réponse à la seconde ligne de traitement. Ceci est dû au fait qu'en se basant sur les scanners, il est impossible d'évaluer la quantité de cellules insensibles au second traitement. Nous sommes convaincus que des données sur la structure de la tumeur, comme des biopsies ou des images fonctionnelles, devraient aider à améliorer la prédictivité du modèle.


\paragraph{}
Dans un troisième chapitre, les aspects numériques seront abordés. 
%En particulier, il a fallu faire face à une instabilité numérique que l'on présentera et à laquelle l'équipe n'avait jamais été confrontée avec les schémas classiques utilisés. %qu'elle utilisait. 
Nous étudions ici des croissances de tumeurs avec traitement, sur du long terme et avec de forts rétrécissements de la taille induit par les traitements. Ce sont là autant de raisons qui présentent l'originalité du problème. Nous montrerons que les champs de vitesses non communs obtenus lors de la simulation numérique  sont à l'origine d'instabilités numériques. 
%Enfin un nouveau schéma de transport sera proposé, le \twinweno, pour palier à cette instabilité.
Enfin une variante du schéma de transport sera proposé, le \twinweno, pour palier à cette instabilité.

\paragraph{}
Dans le quatrième chapitre, nous détaillerons la reconstruction %comment construit-on 
d'images scanners de synthèse à partir des résultats numériques. La simulation numérique fournit la répartition spatio-temporelle de différentes populations cellulaires. En aucun cas cela ne fournit les niveaux d'absorptions des tissus au rayons X, comme mesurés par les scanners. Il nous faut donc un moyen de reconstituer des images scanners de synthèse à partir de ces densités de populations, dans le but de rapprocher au plus près %(une optimisation a été faite) 
le format des résultats numériques à celui des données cliniques.
Cette construction d'images scanners de synthèse permettra une meilleure comparaison des résultats obtenus avec les données cliniques. 
%Elle est également d'un intérêt non négligeable pour le clinicien, qui peut ainsi regarder nos résultats numériques comme s'il s'agissait de scanners réels. 


\paragraph{}
Dans le cinquième et dernier chapitre, nous chercherons un critère permettant de quantifier l'\hetero.  %%\todo[noline]{Etat de l'art sur la construction de quantificateur de l'\hetero} 
%\cite{materka1998texture,castellano2004texture} %% presentation des diverses methodes
%\cite{davnall2012assessment} %%% quantifying heterogeneity by several method
%\cite{haralick1979statistical} %% pere des methodes structurelles et statistiques
%\cite{galloway1975texture,chu1990use,xu2004run} %% run-length matrix
%\cite{pal1993review,hatt2009fuzzy,hatt2010accurate} %%% methode conduisant a une segmentation
%\cite{chabat2003obstructive} %%% classification de texture
%\cite{longo2015cluster} %% methode de clustering
%\cite{ahmed2013texture} %% matrice de cooccurence
% \cite{morris2006using} %% ref julien sur wavelet
De nombreuses possibilités sont offertes~\cite{materka1998texture,davnall2012assessment,castellano2004texture} pour comparer des textures~\cite{chabat2003obstructive,napel2010automated} ou bien détecter des contours ou des zones homogènes~\cite{pal1993review}. On pourra citer notamment des méthodes statistiques~\cite{haralick1979statistical} (étude de diverses quantités provenant des histogrammes de niveaux de gris ou bien de diverses autres quantités comme le gradient absolu ou la matrice run-length~\cite{galloway1975texture,chu1990use,xu2004run} ou encore la matrice de co-occurrence~\cite{ahmed2013texture}), les méthodes de clustering~\cite{longo2015cluster}, les méthodes de segmentation~\cite{hatt2009fuzzy,hatt2010accurate} ou encore les méthodes dites  wavelet~\cite{davnall2012assessment,morris2006using}.
Le critère que l'on utilise dans ce mémoire est 
%Ce critère devra être 
applicable aussi bien aux données cliniques qu'aux résultats fournis par la simulation numérique. 
Et c'est ce qui fait ici l'originalité de la construction de celui-ci. % d'un tel critère. 
La méthode proposée dans ce manuscrit est basée sur l'analyse %des niveaux de gris 
de l'intensité des pixels 
qui constituent les images (cliniques et numériques). 
%A partir de l'histogramme des niveaux de gris, deux dominantes sont identifiées sous forme gaussienne pour chacune des images. 
%A partir de ces deux composantes, un quantificateur de l'\hetero sera construit. 
Après avoir identifié la tumeur sur les images, l'histogramme des niveaux de gris est étudié et décrit à l'aide 
%%de deux gaussiennes. 
d'un modèle de mélange de gaussiennes. 
Le critère, basé sur une évaluation de la proximité des %deux 
composantes gaussiennes, va ainsi traduire le caractère \heterogene de la métastase. 
%%L'idée repose sur le fait que plus les composantes sont éloignées, plus elles traduisent une \hetero. 
%Le critère
Il sera construit en se basant sur les scanners du premier patient (\Nber) uniquement. 
Le second patient (\Chen) sera gardé à titre de vérification, pour valider le critère. 
Une fois le critère construit, il sera appliqué également aux images produites par la simulation numérique. 
La robustesse du critère vis-à-vis de la paramétrisation choisie pour la synthèse des images  numériques sera présentée. 
Ce critère montrera notamment les limites du modèle EDP en terme de reproduction de l'\hetero tumorale.

\paragraph{}
Enfin diverses annexes viendront compléter les discussions menées au cours de ce manuscrit. En particulier, j'attire l'attention sur l'annexe~\ref{chap:anx_img_complement} qui présente notamment, en images, l'ensemble des données cliniques dont nous disposons pour les deux patients, données qui ont toutes été acquises sur la même machine. 

\end{document}