%\documentclass[12pt,a4paper,twoside]{memoir}
\documentclass[a4paper,twoside,final]{book} %%% twoside pour final
%\documentclass[a4paper,oneside,draft]{book} %%% draft pour voir tout ce qui depasse dans les marges
%%\documentclass[a4paper,oneside]{book}

\usepackage[utf8]{inputenc}
\usepackage[francais,frenchb]{babel}
\usepackage[T1]{fontenc}
\usepackage{amsmath,amsfonts,amssymb}
\usepackage{amsthm} %% \newtheorem
\usepackage{graphicx}

\usepackage{lmodern} %% latin modern font
%\usepackage{fcursive}
%%\usepackage[mathscr]{euscript}
\usepackage{euscript}
\usepackage{dsfont}
\usepackage{mathrsfs} %% fournit \mathsrc (majuscules rondes)
\DeclareMathAlphabet{\mathpzc}{OT1}{pzc}{m}{it}
\DeclareMathAlphabet{\matheus}{U}{eus}{m}{n}
\DeclareTextFontCommand{\textpzc}{\fontfamily{pzc}\fontseries{m}\fontshape{n}\selectfont}

%\usepackage{scrextend} %%% http://tex.stackexchange.com/questions/5339/how-to-specify-font-size-less-than-10pt-or-more-than-12pt
%\changefontsizes[20pt]{16pt}
\usepackage[fontsize=13pt]{scrextend}
%\usepackage{frcursive} %fournit l'environnement cursive, qui imite une écriture manuelle
%%%\usepackage{microtype} %% Permet de reduire l'espace entre les caracteres
\usepackage{relsize}
\usepackage{lettrine}

\usepackage[titles]{tocloft}
\setlength{\cftchapnumwidth}{0pt}
\setlength{\cftbeforechapskip}{\baselineskip}
\renewcommand{\cftchapaftersnum}{.}
\renewcommand{\cftchapaftersnumb}{\newline}
%%\renewcommand{\cftchappresnum}{\chaptername\ }
%\makeatletter%
%\ifx\@chapapp\appendixname%
%\renewcommand{\cftchappresnum}{Annexe\ }%
%\else%
%\renewcommand{\cftchappresnum}{\chaptername\ }%
%\fi%
%\makeatother%
\makeatletter
\newcommand*\updatechaptername{%
        \addtocontents{toc}{\protect\renewcommand*\protect\cftchappresnum{\@chapapp\ }}%
}
\makeatother


%%%\usepackage[titletoc,title]{appendix}
%\usepackage[toc,page]{appendix} %% option toc quio ne marche pas --> appel manuel a \addappheadtotoc
\usepackage[page]{appendix}
\renewcommand{\appendixtocname}{{\fontfamily{pzc}\selectfont \Large  Annexes}}
%\renewcommand{\appendixpagename}{Annexes}
\def\appendixpage{\newpage\vspace*{9cm}
\begin{center}
%\Huge\textbf{Annexes}
%\textpzc{\fontsize{2cm}{1em}\selectfont ANNEXES}
\textpzc{\fontsize{3cm}{1em}\selectfont Annexes}
\end{center}
%%$$\Huge\mathcal{Annexes}$$
%$$\Huge\mathpzc{\sc Annexes}$$
\thispagestyle{empty}
}

%%%\usepackage{afterpage}%  %%% Evite le comportement grande figure sur une seule et unique page
\usepackage{float}% Fournit l'option H pour les figures

\usepackage{mathabx} %% Pour avoir proprement le grand symbole \sqrt notamment
\usepackage{mathtools}
\usepackage{caption} %% Permet d'utiliser \captionof sans etre dans un environnement figure
\usepackage{subfig}%% \subfloat
\usepackage{wrapfig}%% \begin{wrapfigure}
%%\usepackage{stackengine}
\usepackage{vwcol} 
\usepackage{placeins} %%% Fournit \FloatBarrier
\usepackage{xspace}
\usepackage{lipsum}
\usepackage{booktabs} %% midrule
\usepackage{array}
\usepackage{multicol}
\usepackage{multirow}
\usepackage{colortbl} %% rowcolor
\usepackage[table]{xcolor} %% gray!40  %% rowcolors
\usepackage{datatool} %%%% See the documentation :
%%%% http://texdoc.net/texmf-dist/doc/latex/datatool/datatool-user.pdf
\usepackage{pict2e}
\usepackage{slashbox}
\usepackage{xfrac}
\usepackage{hhline}
\usepackage{spreadtab}
\usepackage{rotating}

\frenchbsetup{StandardLists=true} %% Resolves conflict between babel and enumitem
\usepackage{enumitem} %%% Personalisation des puces pour les listes 
%%\usepackage{enumerate}

\usepackage{tabularx}
\usepackage{tablefootnote}
%%\usepackage{xcolor} (repetition)
%%\usepackage{showkeys} %%% Pour faire afficher les label utilisés
%%%%%\usepackage[plain]{fancyref}
\usepackage{xr}
\usepackage{hyperref} %%% Les premieres compilations fournisse une erreur. recompiler plsrs fois ...

%%% Dimension page pour final
%\usepackage[left=35mm,right=25mm,top=30mm,bottom=30mm]{geometry} 
%%% Dimension page pour todonote a droite
\usepackage[left=40mm,right=20mm,top=30mm,bottom=30mm]{geometry}
\usepackage[textwidth=30mm]{todonotes} %%% Ajuster le textwidth en fct de la marge
%\usepackage[disable]{todonotes}
\reversemarginpar %%% pour que les notes alterne au niveau des marges (mode twoside)
\setlength{\marginparwidth}{3cm}%%% pour regler la posiition des notes sur page impaire

\usepackage{tcolorbox}
\usepackage{subfiles}
%%\usepackage{etoolbox}
\usepackage{tikz,tkz-tab}
\usepackage{pgfplots}

%\usepackage{titletoc}
%\dottedcontents{subsection}[9em]{}{3em}{.5pc}
%\dottedcontents{section}[6.5em]{}{2em}{.5pc}
%\dottedcontents{chapter}[4em]{\vspace{7mm}}{2em}{.5pc}


%%%%% fncychap predefini
%\usepackage[Glenn]{fncychap}
%\usepackage[Conny]{fncychap}
\usepackage[Bjornstrup]{fncychap}
%\usepackage[Peters]{fncychap}
%\usepackage{fncychap}
%%\usepackage{fancyhdr} %%%% incompatible avec la class memoir
\usepackage{fancyhdr}

%%%%%% Parametrage des entetes et pied de page %%%%%%%%
\pagestyle{fancy}
%\fancyhead{}% efface le contenu de l'en-tete
%\fancyfoot{}% efface le contenu du pied de page

\renewcommand{\chaptermark}[1]{\markboth{{\sc \chaptername } \thechapter.\ #1}{}}
\renewcommand{\sectionmark}[1]{\markright{\thesection.\ #1}}
\addtolength{\footskip}{7mm}%
 
%\lhead[]{\leftmark}
%\rhead[\rightmark]{}
%\lfoot[Guillaume Lefebvre]{\thepage}
%\rfoot[\thepage]{Guillaume Lefebvre}
%%\renewcommand{\headheight}{15pt}
%%\renewcommand{\headsep}{16pt}
%%\renewcommand{\footskip}{40pt}
%%\makeatletter
%%\let\ps@plain=\ps@fancy
%%\makeatother
%\fancypagestyle{plain}{
%\lhead[]{}
%\renewcommand{\headrulewidth}{0pt}%
%}

\fancypagestyle{main}{%
  \fancyhf{}%
  \lhead[\leftmark]{}
  \rhead[]{\rightmark}
  \fancyfoot[LE,RO]{{\normalsize \thepage}}%
  \fancyfoot[LO,RE]{{\footnotesize Guillaume Lefebvre}}%
  \renewcommand{\headrulewidth}{0.4pt}%
}

\fancypagestyle{plain}{%
  \fancyhf{}%
%  \lhead[\leftmark]{}
%  \rhead[]{\rightmark}
  \fancyfoot[LE,RO]{{\normalsize \thepage}}%
  \fancyfoot[LO,RE]{{\footnotesize Guillaume Lefebvre}}%
  %%\addtolength{\footskip}{7mm}%
  \renewcommand{\headrulewidth}{0pt}%
}

\fancypagestyle{prefacestyle}{%
  \fancyhf{}%
%  \lhead[\leftmark]{}
%  \rhead[]{\rightmark}
  \fancyhead[LE,RO]{~\chaptername~}
  \fancyfoot[LE,RO]{{\normalsize \thepage}}%
  \fancyfoot[LO,RE]{{\footnotesize Guillaume Lefebvre}}%
  %%\addtolength{\footskip}{7mm}%
  \renewcommand{\headrulewidth}{0.4pt}%
}

\appto\frontmatter{\pagestyle{prefacestyle}}
%\appto\frontmatter{\pagestyle{main}}
\appto\mainmatter{\pagestyle{main}}
\appto\backmatter{\pagestyle{main}}




%%%%%%% Variables d'environnement
\graphicspath{{../img/}{../img_publi_GIST/}} %%


\setlength\doublerulesep{2mm} %% hauteur entre 2 \hline successifs
\newlength{\retraittableau}
\setlength{\retraittableau}{-4mm}
\newlength{\firstcolwidth} %% on ne peut mettre ni chiffre ni caractere special dans le nom d'une longueur
\setlength{\firstcolwidth}{22mm}
\newlength{\defaultcolumnsep}
\setlength{\defaultcolumnsep}{25pt}
\setlength\columnsep{\defaultcolumnsep}

%\setlist[enumerate,1]{label=\arabic*)}
%\setlist[enumerate,2]{label=\alph*.}
%\setlist[itemize,1]{label=$\circ$}
\newenvironment{myitemize}[1][]{%
\itemize[label=\textbullet ,noitemsep, topsep=0pt,partopsep=0pt, #1 ]%
}{\enditemize}
\newenvironment{myenumerate}[1][]{%
\enumerate[label=\arabic*. ,noitemsep, topsep=.5\topsep, #1 ]%
}{\endenumerate}

\theoremstyle{definition}
\newtheorem{thm}{Théorème}[section]
\newtheorem{dfn}[thm]{Définition}
\newtheorem{prop}[thm]{Propriété}
\newtheorem{lemme}[thm]{Lemme}
\newtheorem{coro}[thm]{Corollaire}

\DeclareMathOperator{\atan}{atan}
\DeclareMathOperator{\atanh}{atanh}

%\setcounter{secnumdepth}{3}
%\setcounter{tocdepth}{3}

%%http://ctan.mines-albi.fr/macros/latex/contrib/lettrine/doc/demo.pdf
\renewcommand{\LettrineFontHook}{
%\fontfamily{yfrak}
\fontfamily{pag}
\fontseries{bx}\fontshape{it}
%\color[gray]{0.5}
}

\renewcommand{\qedsymbol}{$\blacksquare$} %%% Personnalisation du symbole de fin de preuve
%%%% Commande jouent le role inverse de \nonumber dans les environnement avec * (align* par ex.)
\newcommand\numberthis{\addtocounter{equation}{1}\tag{\theequation}}

%%%%%%% Personnalisation des legendes
%\captionsetup[figure]{labelfont=it,textfont={bf,it}}
%\captionsetup[subfigure]{labelfont=bf,textfont=normalfont,justification=raggedright}

\newcommand{\mylettrine}[2]{\lettrine[lines=2, lhang=0.33, loversize=0.25]{#1}{#2}} 
%\newcommand{\mylettrine}[2]{\lettrine[lines=2, lhang=0.33, loversize=0.25]{$\mathscr{#1}$}{#2}} 

\newcommand{\custompart}[3]{
%%%% A voir : \setpartpreamble[taille_facultative]{texte du preambule} du package Koma-script
%%% Fournit aussi : 
%% \renewcommand*{\dictumauthorformat}[1]{\color{blue} \bfseries #1}
%% \dictum[auteur de l’aphorisme]{le texte de l’aphorisme}
%% cf. http://bertrandmasson.free.fr/index.php?categorie5/latex-koma-script
\part{ 
#1 \\ {
% \bf %% bold
% \it %% italic
%\tt %% typewriter style
%\sl %% slanted typestyle
%\rm %%% roman style
%\sc %%% small caps
\sf %% sans serif
\LARGE #2 }
\\ \vspace{3cm}
%\newpage
%\begin{tabular}{p{130mm}}
\normalsize \normalfont #3
%\end{tabular}
}
%\cleardoublepage
}


\newcommand{\specialchap}[1]{%
%%%% Chapitre sans sections
\chapter{#1}%
\pagestyle{prefacestyle}%
\renewcommand\chaptername{#1}%
}

\makeatletter
\newcommand{\leqnomode}{\tagsleft@true}
\newcommand{\reqnomode}{\tagsleft@false}
\makeatother

%%%%%% MES MACROS %%%%%%%%

\newcommand{\hetero}{hétérogénéité\xspace}
\newcommand{\Hetero}{Hétérogénéité\xspace}
\newcommand{\heteros}{hétérogénéité\xspace}
\newcommand{\Heteros}{Hétérogénéité\xspace}
\newcommand{\heterogene}{hétérogène\xspace}
\newcommand{\heterogenes}{hétérogènes\xspace}
\newcommand{\Nber}{Patient~A\xspace}
\newcommand{\Chen}{Patient~B\xspace}
\newcommand{\cad}{c'est-à-dire\xspace}
\newcommand{\ie}{{\it i.e.}\xspace}
\newcommand{\cf}{{\it cf.}\xspace}
\newcommand{\etal}{{\it et al.}\xspace}

\newcommand{\intx}{\int_\Omega}
\newcommand{\intbord}{\int_{\partial \Omega}}
\newcommand{\dvecx}{\;{\rm d}\vecx\xspace}
\newcommand{\dx}{\;{\rm d}x\xspace}
\newcommand{\dy}{\;{\rm d}y\xspace}
\newcommand{\dt}{\;{\rm d}t\xspace}
\newcommand{\du}{\;{\rm d}u\xspace}
\newcommand{\dv}{\;{\rm d}v\xspace}
\newcommand{\vol}{\mathcal{V}}
\newcommand{\aire}{\mathcal{A}}
\newcommand{\intperso}[1]{\!\!\int\limits_{#1}\!\!\!}
%%\newcommand{\trans}{\,^t}
\newcommand{\trans}[1]{\prescript{\mathit{t\!}}{}{#1}}
\newcommand{\ssi}{si et seulement si\xspace}

\newcommand{\CFL}{{\mathrm{CFL}}}
\newcommand{\Id}{{\mathds 1}}

\newcommand{\W}{{\bf W}}

\newcommand{\Frm}{{\rm F}}
\newcommand{\Grm}{{\rm G}}
\newcommand{\Hrm}{{\rm H}}
\newcommand{\crm}{{\rm c}}

\newcommand{\reel}{\mathbb{R}}

%\newcommand{\HH}{~$\mathcal{H}$\xspace}
%\newcommand{\HHmath}{\mathcal{H}\xspace}
%\newcommand{\HH}{~$\mathscr{H}$\xspace}
%\newcommand{\HHmath}{\mathscr{H}\xspace}
\newcommand{\HH}{~\ensuremath{\mathscr{H}}\xspace}
\newcommand{\HHobj}{~\ensuremath{\mathscr{H}_{obj}}\xspace}

\newcommand{\treat}{\mathcal{T}}
\newcommand{\TI}{{\mathcal T_1}}
\newcommand{\TS}{{\mathcal T_2}}
\newcommand{\Tini}[1]{\rm{T}_{ini}^#1}
\newcommand{\Tend}[1]{\rm{T}_{end}^#1}
\newcommand{\chii}{\chi_{\rm i}}
\newcommand{\chiI}{\chi_1}
\newcommand{\chiS}{\chi_2}
\newcommand{\dive}{\nabla\cdot}

\newcommand{\Amin}{\mathcal{A}_{\rm min}}

\newcommand{\TPFS}{{\rm T_{PFS}}}
\newcommand{\Td}{{\rm T_{double}}}

\newcommand{\muI}{\mu_1}
\newcommand{\muS}{\mu_2}
\newcommand{\nuS}{\nu_2}
\newcommand{\muN}{\delta}

\newcommand{\RR}{{\mathbb{R}}}

\newcommand{\nn}{{\bf n}}
\newcommand{\GG}{{\bf G}}

\newcommand{\ang}{{\mathfrak{a}}}
\newcommand{\ex}{{\bf e}_x}
\newcommand{\ey}{{\bf e}_y}

\newcommand{\er}{{\bf e}_r}
\newcommand{\et}{{\bf e}_{\theta}}

\newcommand{\gammapp}{\gamma_{pp}}
\newcommand{\gammapd}{\gamma_{pd}}
\newcommand{\gammasd}{\gamma_{sd}}
\newcommand{\Ms}{M_{th}}
\newcommand{\vit}{\mathbf{v}}
\newcommand{\vecx}{\mathbf{x}}
\newcommand{\vecy}{\mathbf{y}}
\newcommand{\Tfini}{{T_f}_{ini}}
\newcommand{\Tgini}{{T_g}_{ini}}
\newcommand{\Tfend}{{T_f}_{end}}
\newcommand{\Tgend}{{T_g}_{end}}
\newcommand{\twinweno}{twin-WENO5\xspace}
\newcommand{\Twinweno}{Twin-WENO5\xspace}

\newcommand{\diff}[2]{\frac{\partial #1}{\partial #2}}

\newcommand{\samefootnote}[1]{\textsuperscript{\hypersetup{hidelinks}\ref{#1}}}
%%\newcommand{\samefootnote}[1]{\footref{#1}} %%% avec le packacke footmisc

\newcommand{\myhrule}{\begin{center}
\hspace*{.05\textwidth}\textsuperscript{\rule{.33\textwidth}{0.4pt}} \hfill 
\begin{tabular}{ccc}&*& \\ *&&*\end{tabular}\hfill 
\textsuperscript{\rule{.33\textwidth}{0.4pt}}\hspace*{.05\textwidth}
\end{center}}

%%%%%% ===================================================
%%%%%%          ***  Commande de dessins   ***
%%%%%% ===================================================

\newcommand{\rect}[4]{
	%% #1 et #2 --> coordonnee du coin en bas a gauche
	%% #3 et #4 --> largeur et hauteur
	\multiput(#1,#2)(0,#4){2}{\line(1,0){#3}}
	\multiput(#1,#2)(#3,0){2}{\line(0,1){#4}}
}

\newcommand{\mesh}[6]{
	%% #1 : x0  , #2 : dx , #3 : Nx (nombre de maille)
	%% #4 : y0  , #5 : dy , #6 : Ny
	%% lignes horizontales
	\put(#1,#4){\multiput(0,#5)(0,#5){#6}{\multiput(0,0)(#2,0){#3}{\line(1,0){#2}}}}
	\multiput(#1,#4)(#2,0){#3}{\line(1,0){#2}}
	%% lignes verticales
	\put(#1,#4){\multiput(#2,0)(#2,0){#3}{\multiput(0,0)(0,#5){#6}{\line(0,1){#5}}}}
	\multiput(#1,#4)(0,#5){#6}{\line(0,1){#5}}
}

\newcommand{\squaremesh}[3]{
	\mesh{#1}{#2}{#3}{#1}{#2}{#3}
}

\newlength{\dimx}
\setlength{\dimx}{888px}
\newlength{\dimy}
\setlength{\dimy}{888px} %%% faire \the\dimy dans le corps du doc pour afficher
\newlength{\zoneh}
\setlength{\zoneh}{90px}
\newlength{\zoneb}
\setlength{\zoneb}{90px}
\newlength{\zoneg}
\setlength{\zoneg}{0px}
\newlength{\zoned}
\setlength{\zoned}{416px}
\newlength{\margh}
\newlength{\margb}
\newlength{\margg}
\newlength{\margd}
\newlength{\dimxn}
\newlength{\dimyn}


\newcommand{\evalNewDim}{%
	\setlength{\dimxn}{\dimx-\margg-\margd}%
	\setlength{\dimyn}{\dimy-\margh-\margb}%
}

\newcommand{\RAZmargin}{%
	\setlength{\margh}{0px}%
	\setlength{\margb}{0px}%
	\setlength{\margg}{0px}%
	\setlength{\margd}{0px}%
	\evalNewDim%
}

\newcommand{\setzone}{%
	\addtolength{\margh}{\zoneh}%
	\addtolength{\margb}{\zoneb}%
	\addtolength{\margg}{\zoneg}%
	\addtolength{\margd}{\zoned}%
	\evalNewDim%
}

\newcommand{\zoomeur}[1]{%
	\evalNewDim%
	\FPeval{\ccc}{0.5*(#1-1.0)/#1}%%% #1 est un facteur d'agrandissement
	\addtolength{\margh}{\ccc\dimyn}%
	\addtolength{\margb}{\ccc\dimyn}%
	\addtolength{\margg}{\ccc\dimxn}%
	\addtolength{\margd}{\ccc\dimxn}%
	\evalNewDim%
}

\newcommand{\decale}[2]{%
	%% #1 > 0 ==> decalage vers la droite
	%% #2 >0 ==> decalage vers le haut
	\addtolength{\margg}{#1}%
	\addtolength{\margd}{-#1}%
	\addtolength{\margh}{#2}%
	\addtolength{\margb}{-#2}%
}

\newcommand{\reshapeimg}[3]{%%% on a pas la mm hauteur si on zoom
	\RAZmargin\setzone\decale{#2}{#3}\zoomeur{#1}%
}

\newcommand{\tikzzoom}[4]{%
  \begin{tikzpicture}[%
  spy using outlines={%
  %%height = .5cm, 
  %%%width = .5cm,
  %%circle,
  magnification=#2,%%% taille de ce que l'on veut zoomer
  size=30mm,%%% taille de la fenetre qui donne affiche le zoom
  connect spies,%
  white}]%
 \node[inner sep=1pt] {%
 \includegraphics[trim = {\margg} {\margb} {\margd} {\margh}, clip, width=0.32\textwidth]{#1}};%
  \spy on (#3,#4) in node at (-0.7,-2.0);%
  %% \spy on (coord de la zone a zoomer) in node at (coord de la vue zoomé )
\end{tikzpicture}%
}

\newcommand{\includeminisimunber}[1]{
\includegraphics[ trim = 51px 20mm 43mm 8mm ,clip, 
width=\largeurfignber]{fit_henbert_form3/vue_scan/vue_scan#1.png}
}

\newcommand{\includeminisimuchen}[1]{
%\resizebox{\largeurfignber}{!}{
%%% trim --- > coupe à gauche, en bas, à droite et en haut
%\setlength{\unitlength}{1mm}
%\begin{picture}(24,24)
%\rect{0}{0}{26}{26}
%\put(0,0){
\includegraphics[ trim = 51px 20mm 43mm 8mm ,clip,
 width=\largeurfignber]{fit_chen7_L12_cross0.3/vue_scan/vue_scan#1.png}
%}
%{\color{white}
%\multiput(30,3)(0,.5){45}{\circle*{1}}
%}
%\end{picture}
%}
}



%%%===============================%%%%
%%%       DEBUT DU DOCUMENT      %%%%
%%%===============================%%%%
\begin{document}
\renewcommand{\contentsname}{Sommaire}
\renewcommand*{\listtablename}{Liste des tableaux}
\renewcommand*{\listfigurename}{Liste des figures}

%%\thispagestyle{empty}
%%\maketitle

\frontmatter % Le prologue du livre
%%\updatechaptername
%%\renewcommand\chaptername{Sommaire}
\pagestyle{plain}
%%\setcounter{tocdepth}{5}
\addcontentsline{toc}{chapter}{Sommaire}
\tableofcontents    % Table des matières

%\specialchap{Avant-propos}
%%\chapter{Avant-propos}
%%\pagestyle{prefacestyle}
%%\renewcommand\chaptername{Avant-Propos}
%\lipsum[1-11]

\specialchap{Introduction}
%\chapter{Introduction}

\mainmatter % Corps du livre
\renewcommand\chaptername{Chapitre}
\updatechaptername

%%%\newcommand{\mainfile}{} % we use the existence of this command to see if we're compiling the whole thesis or just a chapter


%{\tiny abcdefghijklmnopqrstuvwxyzABCDEFGHIJKLMNOPQRSTUVWXZ}\\
%{\scriptsize abcdefghijklmnopqrstuvwxyzABCDEFGHIJKLMNOPQRSTUVWXZ}\\
%{\footnotesize abcdefghijklmnopqrstuvwxyzABCDEFGHIJKLMNOPQRSTUVWXZ}\\
%{\small abcdefghijklmnopqrstuvwxyzABCDEFGHIJKLMNOPQRSTUVWXZ}\\
%{\normalsize abcdefghijklmnopqrstuvwxyzABCDEFGHIJKLMNOPQRSTUVWXZ}\\
%{\large abcdefghijklmnopqrstuvwxyzABCDEFGHIJKLMNOPQRSTUVWXZ}\\
%{\Large abcdefghijklmnopqrstuvwxyzABCDEFGHIJKLMNOPQRSTUVWXZ}\\
%{\LARGE abcdefghijklmnopqrstuvwxyzABCDEFGHIJKLMNOPQRSTUVWXZ}\\
%{\huge abcdefghijklmnopqrstuvwxyzABCDEFGHIJKLMNOPQRSTUVWXZ}\\
%{\Huge abcdefghijklmnopqrstuvwxyzABCDEFGHIJKLMNOPQRSTUVWXZ}\\
%\lipsum[1]

%\custompart{Modèles EDPs de croissance tumorale}{Cas des métastases hépatiques}{Description de la partie.... }

%\subfile{chap1}
\subfile{chap_biologie_du_cancer}
\subfile{chap_modeleEDP}
\subfile{chap_trefle}
\subfile{chap_optim_grey}
\subfile{chap_analyse_heterogeneite}

\appendix %%%% Les annexes
\addappheadtotoc %%% Pour la page debutant les annexes apparaisse dans la toc
\updatechaptername
%%\renewcommand\thechapter{\appendixname~\Alph{chapter}}
\renewcommand\chaptername{Annexe}
%%\renewcommand\chaptername{}
%\renewcommand*{\cftchappresnum}{Annexe~}
\begin{appendices}
\subfile{annexe_tableaux_graphiques_complementaires}
\subfile{annexe_schema_mixte_EFVF}
\subfile{annexe_penalisation}
\subfile{annexe_gaussienne}
\end{appendices}

\backmatter  %%%% l'épilogue du document
\specialchap{Test}
%\updatechaptername
%\renewcommand\chaptername{Test}
%\chapter{Test} %%%pour allonger le sommaire (pour faire des tests)
\subsubsection{test1}\paragraph{hep}
\subsubsection{test2}\paragraph{hep}
\subsubsection{test3}\paragraph{hep}
\subsubsection{test4}\paragraph{hep}
\subsubsection{test5}\paragraph{hep}
\subsubsection{test6}\paragraph{hep}
\subsubsection{test7}\paragraph{hep}
\subsubsection{test8}\paragraph{hep}
\subsubsection{test9}\paragraph{hep}
\subsubsection{test10}\paragraph{hep}
\subsubsection{test11}\paragraph{hep}
\subsubsection{test12}\paragraph{hep}
\subsubsection{test13}\paragraph{hep}
\subsubsection{test14}\paragraph{hep}

\specialchap{Remerciements} %%% defauts si mis dans un autre fichier avec subfile
%\chapter{Remerciements}
%\renewcommand\chaptername{Remerciements}
Je tiens à remercier ici l'ensemble des personnes qui m'ont aidé dans mes travaux de thèse et la réalisation de ce mémoire.

\paragraph{}
En premier lieu, je remercie Thierry Colin, professeur de mathématiques de l'Institut Polytechnique de Bordeaux et directeur du cluster d'excellence CPU de l'Université de Bordeaux. Tout d'abord en tant que directeur de stage de découverte de première année de master. Il m'a initié au développement de modèles mathématiques pour la cancérologie et ce fut une très bonne expérience scientifique et humaine. Ensuite, en tant que directeur de thèse, il m'a guidé dans mon travail tout au long de ces 3 ans et m'a aidé à trouver des solutions pour avancer (ou pour ne pas perdre son temps). 

\paragraph{}
Je remercie aussi Olivier Saut, directeur %chargé 
de recherche CNRS, qui m'a épaulé au cours de ces 3 années également. Il m'a particulièrement aidé sur l'aspect numérique, logiciel et informatique de ma thèse. Je lui suis reconnaissant pour sa patience lors de la résolution des différents problèmes informatiques que j'ai pu rencontrer. 

\paragraph{}
Je remercie également Clair Poignard, chargé de recherche INRIA, pour ses lectures et relectures minutieuses et ses corrections effectuées sur l'ensemble de mes productions écrites~: publications et ce manuscrit entre autres.

\paragraph{}
Je remercie François Cornelis, radiologue du CHU de Bordeaux, pour sa collaboration. Merci à lui d'avoir pris le temps de répondre à l'ensemble de mes interrogations d'ordre clinique et biologique sur les mécanismes des cancers (et plus particulièrement du GIST et des métastases hépatiques), des traitements et de l'imagerie notamment. Les discussions avec François ont toujours été très enrichissantes pour moi. 

\paragraph{}
Je tiens à remercier Stéphanie Salmon et Simona Mancini d'avoir accepté de rapporter mon manuscrit de thèse.\todo{remerciements jury}


\paragraph{}
Je remercie Hassan Fathallah-Shaykh, professeur à l'université de l'Alabama à Birmingham, pour le temps qu'il m'a consacré. En particulier, sa relecture de ma publication a particulièrement contribué  à l'amélioration de celle-ci.

\paragraph{}
Je remercie également Patricio Cumsille, chercheur de l'université du Biobío (Chili) et de l'université du Chili (Santiago, Chili), pour sa collaboration. Il a étudié avec moi, lors de sa visite d'un an en France, au sein de notre équipe de recherche, un modèle qui a précédé celui présenté dans ce manuscrit. 

\paragraph{}
Je remercie aussi, l'ensemble des personnes avec qui j'ai pu tour à tour partager un bureau. Michaël Leguèbe, Julie Joie, Manon Deville, Thibaut Kritter et Guillaume Dechristé notamment, pour l'intérêt porté à mes questions ouvertes et les réponses qu'ils m'ont apportées.

\paragraph{}
Je remercie également l'ensemble des doctorants de l'équipe MC2\footnote{Modelling, Control and Computations}, récemment divisée en deux nouvelles équipes (MEMPHIS\footnote{Modeling Enablers for Multi-PHysics and InteractionS} et MONC\footnote{Modélisation Mathématique pour l'Oncologie}), qui au fil du temps sont devenus bien plus que de simples collègues, en particulier Hervé Ung, Alexia de Brauer, Thomas Michel, Etienne Baratchart et Alice Raeli. Les divers débats abordés et expériences partagées avec eux, scientifiques ou non d'ailleurs, ont été une occasion de se cultiver toujours un peu plus.

\paragraph{}
Je remercie également l'ensemble des bonnes volontés qui ont consacré de leur temps pour relire tout ou partie de mon manuscrit, et d'en avoir décelé les fautes et coquilles en tout genre. 
Merci notamment à Vinciane, Cynthia et Eléonore notamment pour leurs contributions à cette tâche non des plus aisées.

\paragraph{}
Enfin, je remercie tout particulièrement Vinciane pour sa bienveillance et son soutien au quotidien. 
Merci de la force et de la motivation que tu me donnes. 
Merci de ta patience et de ta compréhension face à mes journées de travail qui se sont allongées et intensifiées durant la période de rédaction du présent manuscrit. 

%%  ========= Financements ==============
\newpage
\section*{Financements / Funding}
Ces travaux ont été financés par l'Université de Bordeaux. 
Merci également à l'INRIA et au CNRS d'avoir financée plusieurs missions en France et à l'étranger.


\paragraph{}
This study was supported by a public grant from the French National Research
Agency within the context of the Investments for the Future Program, referenced ANR-10-LABX-57 and named TRAIL
and with the financial support from the French State, managed by the French National Research Agency (ANR) in the frame of the "Investments for the future" Programme IdEx Bordeaux - CPU (ANR-10-IDEX-03-02).


\paragraph{}
Experiments presented in this paper were carried out using the PlaFRIM experimental testbed,
being developed under the Inria DIHPES development action with support from LABRI and
IMB and other entities: Conseil Régional d’Aquitaine, FeDER, Université de Bordeaux and
CNRS (see \url{https://plafrim.bordeaux.inria.fr/})


%%\chapter{Conclusion et discussion}

% Les différentes tables
%\tableofcontents    % Table des matières
%\listoffigures        % Liste des figures
%\listoftables        % Liste des tableaux
%------------ Acronyms List ----------------------------------
%\thispagestyle{fancyplain}
%\chapter*{List of Acronyms}
%\addcontentsline{toc}{chapter}{List of Acronyms}


%%%% Bibliographie
%\nocite{*}
%\bibliographystyle{plain}
%\bibliography{biblio}
\end{document}
